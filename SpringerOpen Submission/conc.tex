%Applications that make heavy use of big data application frameworks require intensive reasoning of the system architecture. 
%We set out to assist the design-time formal verification of big data designs by 

This paper proposes an approach allowing designers and developers to perform analysis of big-data applications by means of code analysis and formal verification techniques.
OSTIA provides support to both in the following sense:
%It provides automated constraint verification in order to identify design anti-patterns and provide structural refactorings. 
it helps designers and developers by recovering the architectural topology on-the-fly from the application code and by assisting them in: 
(a) reasoning on the topological structure and how to refine it; 
(b) exporting the topological structure consistently with restrictions of their reference development framework so that further analysis (e.g., formal verification) may ensue. In addition, while performing on-the-fly architecture recovery, the analyses focuses on checking for the compliance to essential consistency rules specific to targeted big data frameworks. 
(c) Finally, OSTIA allows designers to check whether the recovered topologies contain occurrences of key anti-patterns. By running a case-study with partner organizations, we observed that OSTIA assists designers and developers in establishing and continuously improving the quality of topologies behind their big data applications. 
%We confirmed this result running OSTIA on several open-source applications featuring streaming technologies.  % \url{https://github.com/maelstromdat/OSTIA}. 

OSTIA can be easily extended to provide more refined tools for the analysis of data-intensive applications as it is general in the approach and modular with respect to the definition of (i) the anti-patterns to be considered and (ii) the formal analysis approaches and the application modeling to be adopted.
For this reason, in addition to the practical evidence observed,  we believe that OSTIA can be considered as a reference point in the development of data-intensive applications.
This motivates us to further elaborate the anti-patterns, 
%that may emerge across big data topologies 
exploiting graphs analysis techniques inherited from social-networks analysis. Also, we plan to expand OSTIA to support technologies beyond the most common application framework for streaming %, i.e., Storm. 
and, finally, to further evaluate OSTIA using empirical evaluation.

%{\small\subsubsection*{Acknowledgment} The Authors' work is partially supported by the European Commission grant no. 644869 (EU H2020), DICE.}

%{\small\subsubsection*{Appendix} Please follow the link to navigate to the appendices: \url{http://tinyurl.com/zco4sdz}.}