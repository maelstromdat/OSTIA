%\textbf{@Marcello,Francesco: we should also probably elaborate on the kind of verification technique we are using and how that can help in evaluating the topology.. remember here we do not have the DICE restriction so we can mention any kind of analysis that it would be possible to run, also analyses that are currently in the hands of other DICE partners!!}

%\begin{itemize}
%\item we can use the ATC case study as much as we want - that yields already three topologies that we can infer
%\item ATC has agreed that we can mention their role in this exercise, I also showed them the topology that we elicited basically with OSTIA and they already made considerations on how to improve it
%\item in the evaluation we should also comment on how OSTIA can help you in visualizing the application topology that you may be considering to use by reusing a big-data application for something else... visualising the application topology and analysing it may allow you to improve it while you are using it as a starting point for your application
%\item another application that we can use is the one that NETF is considering for their own scenario, KILLRWEATHER - \url{https://github.com/killrweather/killrweather}
%\item any additional case that we can run?
%\item what do the results show? do we have a way to quickly quantify the time that is saved by using this approach? e.g., the time that is saved in setting up and running the infrastructure and how much would that time saved have costed these could be valuable evaluation insights
%\end{itemize}
We evaluated OSTIA through qualitative evaluation and case-study research featuring an open-/closed-source industrial case study (see Section \ref{cs}) and two open-source case studies (see Section \ref{os}) on which we also applied OSTIA-based formal verification and refactoring (see Section \ref{ver}). The objective of the evaluation was two-fold:
\begin{enumerate}
\item[OBJ.1] Evaluate the occurrence of anti-patterns evidenced by our practitioners in both open- and closed-source DIAs;
\item[OBJ.2] understand whether OSTIA-based analyses aid in refactoring towards formally-verified DIA topologies \emph{by-design};
\end{enumerate}

\subsection{Establishing Anti-Patterns Occurrence with Case-Study Research: 3 Cases from Industry}\label{cs}

%As previously introduced in Section \ref{ra}, 
OSTIA was evaluated using 3 medium/large topologies (11+ elements) part of the SocialSensor App. Our industrial partner is having
performance and availability outages connected to currently unknown
circumstances. Therefore, the objective of our evaluation for OSTIA was twofold:
(a) allow our industrial partner to enact architecture refactoring of their
application with the goal of discovering any patterns or hotspots that may be
requiring further architectural reasoning; (b) understand whether OSTIA provided
valuable feedback helping designers in tuning their application through a design-and-refactor loop.%to endure the continuous architecting exercise.


In addition to formal verification, specific algorithms for graph analysis can be integrated in OSTIA to offer a deeper insight of the applications.
For instance, the industrial case study has been analyzed with two algorithms to identify linear sequences of nodes and clusters in the topology graph.
Topology linearisation results in sorting the processing elements in a topology in a way that topology looks more linear, visually. 
This step ensures that visual investigation and evaluation of the structural complexity of the topology is possible by direct observation. 
Topology clustering implies identifying coupled processing elements (i.e., bolts and spouts) and cluster them together (e.g., by means of graph-based analysis) in a way that elements in a cluster have high cohesion and loose-coupling with elements in other clusters. 
Simple clustering or Social-Network Analysis mechanisms can be used to infer clusters. 
Clusters may require, in general, additional attention since they could turn out to become bottlenecks. 
Reasoning more deeply on clusters and their resolution may lead to establishing the Storm scheduling policy best-fitting with the application.


OSTIA standard output\footnote{Output of OSTIA analyses is not shown fully for the
sake of space.} for the smallest of the three SocialSensor topologies, namely
the ``focused-crawler" topology, is outlined in Fig. \ref{topo1}.

\begin{figure*}
\begin{center}
		\includegraphics[width=11cm,draft]{images/output/fig7}
		\caption{SocialSensor App, OSTIA sample output partially linearised (top) and cascaded (bottom left and right).}
		\label{topo1}
		\end{center}
\end{figure*}



%OSTIA has been proved particularly helpful in visualising the complex topologytogether with the parallelism level of each components. 


%REMOVED MOVED TO CASE-STUDY ANALYSIS 
%\subsubsection{Topology clustering}\label{3}
%Topology clustering is outlined in Fig. \ref{fig:clustering}. Topology clustering implies identifying coupled processing elements (i.e., bolts and spouts) and cluster them together (e.g., by means of graph-based analysis) in a way that elements in a cluster have high cohesion and loose-coupling with elements in other clusters. Simple clustering or Social-Network Analysis mechanisms can be used to infer clusters. These clusters may require additional attention since they could turn out to become bottlenecks. Reasoning more deeply on clusters and their resolution may lead to establishing the Storm scheduling policy best-fitting with the application. We will elaborate on this in Section \ref{sec:performance-boosting}.
%%\emph{\bf Does it relates with Storm scheduling?}
%
%\begin{figure}
%	\begin{center}
%		\includegraphics[width=6.5cm]{images/clustering}
%		\caption{clustering.}
%		\label{fig:clustering}
%	\end{center}
%\end{figure}


%REMOVED MOVED TO CASE-STUDY ANALYSIS 
%\subsubsection{Linearising a topology}\label{4}
%
%Topology linearisation is outlined in Fig. \ref{fig:linearizing}. Sorting the processing elements in a topology in a way that topology looks more linear, visually. This step ensures that visual investigation and evaluation of the structural complexity of the topology is possible by direct observation. It is sometimes essential to provide such a visualisation to evaluate how to refactor the topology as needed.

%\begin{figure}
%	\begin{center}
%		\includegraphics[width=5cm]{images/linearizing}
%		\caption{linearising.}
%		\label{fig:linearizing}
%	\end{center}
%\end{figure}



Combining this
information with runtime data (i.e., latency times) our industrial partner observed
that the ``expander" bolt needed additional architectural reasoning. More in particular, the bolt in question concentrates a lot of the topology's progress on its queue, greatly hampering the topology's scalability. In our partner's scenario, the limited scalability was blocking the expansion of the topology in question with more data sources and sinks.
In addition, the partner welcomed the idea of using OSTIA as a mechanism to enact the refactoring of the topology in question as part of the needed architectural
reasoning.
%\comment{for example? elaborate more on this. the reviewers will likely say its vague...}

%Besides this pattern-based evaluation and assessment, OSTIA algorithmic analyses\todoMB{}{Non mi è chiaro quali siano le analisi oltre ai pattern...}
OSTIA assisted our client in understanding that the topological structure of the
SocialSensor app would be better fit for batch processing rather than streaming,
since the partner observed autonomously that too many database-output spouts and
bolts were used in their versions of the SocialSensor topologies. In so doing,
the partner is now using OSTIA to drive the refactoring exercise towards a
Hadoop Map Reduce~\cite{hadoop}
%\footnote{\url{http://hadoop.apache.org/}} 
framework for batch processing.

%\comment{this section is not good enough, we require to add more information, for example what would be the target architecture after refactoring looks like? we said some patters were idenfitified but we didnt say details of continuous rearchitecting...and also formal verification process for this}

\begin{figure}
\begin{center}
\includegraphics[width=8cm,draft]{images/fig8}
		\caption{Industrial case-study, a refactored architecture.}
		\label{atc}
		\end{center}
\end{figure}

As a followup of our analysis, our partner is refactoring his own high-level software architecture adopting a lambda-like software architecture style \cite{lambda} (see Fig. \ref{atc}) which includes the Social-Sensor App (Top of Fig. \ref{atc}) as well as several additional computation components. In summary, the refactoring resulting from OSTIA-based analysis equated to deferring part of the computations originally intended in the expander bolt within the Social Sensor app to additional ad-hoc Hadoop Map Reduce jobs with similar purpose (e.g., the EntityExtractor compute node in Fig. \ref{atc}) and intents but batched out of the topological processing in Storm (see Fig. \ref{atc})\footnote{several other overburdened topological elements were refactored but were omitted here due to industrial secrecy}. 

Our qualitative evaluation of the refactored architecture by means of several interviews and workshops revealed very encouraging results. 
%However, we are yet to quantitatively evaluate whether the new software architecture actually reflects a tangible boost in terms of performance and scalability.\todoMB{}{Azz...pesante questa...}

\subsection{Establishing Anti-Patterns Occurrence with Case-Study Research: 3 Cases from Open-Source}\label{os}

To confirm the usefulness and capacity of OSTIA to enact a refactoring cycle, we applied it in understanding (first) and attempting
improvements of two open-source applications, namely, the previously introduced
DigitalPebble~\cite{digitalpebble} and 
%\footnote{\url{https://github.com/DigitalPebble}} and
StormCV~\cite{stormCV}
%\footnote{\url{https://github.com/sensorstorm/StormCV}}
applications. Figures \ref{dp} and \ref{scv} outline standard OSTIA output for the two applications. Note that we did not have any prior knowledge concerning the two applications in question and we merely run OSTIA on the applications' codebase dump in our own experimental machine. OSTIA output takes mere seconds for small to medium-sized topologies (e.g., around 25 nodes). 
%
\begin{figure}
\begin{center}
\includegraphics[width=4cm,draft]{images/output/fig9}
		\caption{StormCV topology (linearised).}
		\label{scv}
		\end{center}
\end{figure}
%%%%
%%%%\begin{figure}
%%%%\label{fig:oscasestudy}
%%%%\centering 
%%%%\subfigure[{\footnotesize DigitalPebble topology.}]{\includegraphics[width=4.5cm]{images/output/crawl}}\label{dp}
%%%%\hspace{0.5cm}
%%%%\subfigure[{\footnotesize StormCV topology.}]{\includegraphics[width=3cm]{images/output/senti_storm}}\label{scv}
%%%%\end{figure}

The OSTIA output aided as follows: (a) the output summarised in Fig. \ref{dp}
allowed us to immediately grasp the functional behavior of the DigitalPebble and
StormCV topologies allowing us to interpret correctly their operations before
reading long documentation or inspecting the code; (b) OSTIA aided us in visually interpreting the complexity of the applications at hand; (c) OSTIA allowed us to spot several anti-patterns in the DigitalPebble Storm application around the ``sitemap" and ``parse" bolts, namely, a multiple cascading instance of the multi-anchoring pattern and a persistent-data pattern. Finally, OSTIA aided in the identification of the computational funnel anti-pattern around the "status" bolt closing the DigitalPebble topology. With this evaluation at hand, developers in the respective communities of DigitalPebble and StormCV could refactor their topologies, e.g., aided by OSTIA-based formal verification that proves the negative effects of said anti-patterns.

\begin{framed}
\textbf{Summary for Obj 1.} The patterns we elicited thanks to focus-groups in industry indeed have an actual recurrent manifestation in both industry and open-source. OSTIA-based analysis can support reasoning and potential refactoring of the proposed anti-patterns.
\end{framed}
%
%\comment{this section needs further elaboration. we need to elaborate our discussion to visually locate these anti patterns.}
% \comment{you said first, where is the second? this is incomplete}
%
%\begin{figure}
%\begin{center}
%		\includegraphics[width=2.7cm]{}
%		\caption{}
%		\label{scv}
%\end{center}
%\end{figure}

\subsection{OSTIA-based Formal Verification and Refactoring}

In this section we outline the results from OSTIA-based formal verification applied on (one of) the topologies used by our industrial partner in practice. 
Results provide valuable insights for improving these topologies through refactoring.

\begin{figure}
\begin{center}
\includegraphics[width=6cm,draft]{images/output/fig10}
		\caption{DigitalPebble topology.}
		\label{dp}
		\end{center}
\end{figure}
The formal analysis of the ``focused-crawler'' topology confirmed the critical role of the ``expander'' bolt, previously noticed with the aim of OSTIA visual output. It emerged from the output traces that there exists an execution of the system, even without failures, where the queue occupation level of the bolt is unbounded. Figure~\ref{verif-trace} shows how the tool constructed a periodic model in which a suffix (highlighted by the gray background) of a finite sequence of events is repeated infinitely many times after a prefix (on white background). After ensuring that the trace is not a spurious model, we concluded that the expander queue, having an increasing trend in the suffix, is unbounded. 
As shown in the the output trace at the bottom of Fig.~\ref{verif-trace}, further analyses on the DigitalPebble use case revealed that the same problem affects the ``status'' bolt of the DigitalPebble topology. This finding from the formal verification tool reinforced the outcome of the anti-pattern module of OSTIA, showing how the presence of the computational funnel anti-pattern could lead to an unbounded growth in the queue of the ``status'' bolt.
These types of heavyweight and powerful analyses are made easier by OSTIA in that our tool provides a ready-made analyzable models of the topologies making almost invisible the formal verification layer (other than manually setting and tuning operational parameters for verification). %on top of which OSTIA support is harnessed.
%
%\comment{discuss thge benefit of this, was it possible without formal verification?  elaborate more}
% alternative
%This feedback persuaded to 

\begin{figure}
\centering
\includegraphics[width=1\linewidth,draft]{images/fig11}
\caption{OSTIA-based formal verification output traces showing the evolution of the two bolts over time. Queue trends are displayed as solid black line. Dashed lines show the processing activity of the bolts, while the other lines illustrate the incoming tuples from the subscribed nodes (\texttt{emit} events).}
\label{verif-trace}
\end{figure}


\begin{framed}
\textbf{Summary for Obj 2.} OSTIA-based formal verification effectively evaluates the safety of DIAs focusing on their design-time representation; further investigation of the generalisability of this approach towards runtime is needed to scope the extent to which OSTIA offers support for continuous evolution.
\end{framed}

