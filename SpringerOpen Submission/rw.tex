%\begin{itemize}
%\item mention DICE
%\item mention work by Len Bass on Big-Data
%\item other stuff on big data?
%\item feel free to extend this section with Previous work of course :)
%\end{itemize}

The work behind OSTIA stems from the EU H2020 Project called DICE~\cite{dice2020}
%\footnote{\url{http://www.dice-h2020.eu/}} 
where we are investigating the use of model-driven facilities to support the design and quality enhancement of big data applications. Much similarly to the DICE effort, the IBM Stream Processing Language (SPL) initiative \cite{ibmspl} provides an implementation language specific to programming streams management (e.g., Storm jobs) and related reactive systems. In addition, there are several work close to OSTIA in terms of their foundations and type of support, e.g., works focusing on distilling and analysing big data topologies \emph{by-design}~\cite{SNASEL2017286}, as also highlighted in recent research by Kalantari et al. \cite{Kalantari2017}. 

First, from a non-functional perspective, much literature discusses quality analyses of Big Data topologies, e.g., from a performance~\cite{perfbd} or reliability point of view \cite{bigdatareliab}. Existing work use complex math-based approaches to evaluating a number of big data architectures, their structure and general configuration. However, these approaches do not suggest any architecture refactorings. With OSTIA, we automatically elicits a Storm topology, analyses the topologies against a number of consistency constraints that make the topology consistent with the framework. To the best of our knowledge, no such tool exists to date. Furthermore, as highlighted by Olshannikova et al. \cite{Olshannikova2015} the few works existing on big data processes and their visualization highlight a considerable shortcoming in tools and technologies to visualize and interact with data-intensive models at runtime \cite{Olshannikova2015}.

Second, from a modelling perspective, approaches such as StormGen~\cite{stormgen} offer means to develop Storm topologies in a model-driven fashion using a combination of generative techniques based on XText and heavyweight (meta-)modelling, based on EMF, the standard Eclipse Modelling Framework Format. Although the first of its kind, StormGen merely allows the specification of a Storm topology, without applying any consistency checks or without offering the possibility to \emph{recover} said topology once it has been developed. By means of OSTIA, designers can work refining their Storm topologies, e.g., as a consequence of verification or failed checks through OSTIA. Tools such as StormGen can be used to assist preliminary development of quick-and-dirty topologies.
%Fourth, from a verification perspective, no previous effort tried yet to combine formal verification and architectural modelling of streaming topologies. Our attempt serves as a first rudimentary effort towards using complex and valuable verification approaches in combination with lightweight and agile DevOps inspired tools and approaches.
%%...\\
%%\textbf{@Marcello,Francesco: here we should probably elaborate on what kind of verification approach we are using and what other verifications may be done, e.g., using some related work at this point... e.g., is there any other verification attempt considering JSON as an interchange format? I would discuss these and compare them to OSTIA as a whole}
%=======
%\textbf{@Marcello,Francesco: here we should probably elaborate on what kind of verification approach we are using and what other verifications may be done, e.g., using some related work at this point... e.g., is there any other verification attempt considering JSON as an interchange format? I would discuss these and compare them to OSTIA as a whole}\\

Third, from a verification perspective, to the best of our knowledge, this represents the first attempt to build a formal model representing Storm topologies, and the first try in making a configurable model aiming at running verification tasks of non-functional properties for big data applications. While some works concentrate on exploiting big data technologies to speedup verification tasks~\cite{camilli2014}, others focus on the formalization of the specific framework, but remain application-independent, and their goal is rather to verify properties of the framework, such as reliability and load balancing~\cite{dicomputational}, or the validity of the messaging flow in MapReduce~\cite{yang2010formalizing}.
%\footnote{The Authors' work is partially supported by the European Commission grant no. 644869 (EU H2020), DICE. Also, Damian's work is partially supported by the European Commission grant no. 610531 (FP7 ICT Call 10), SeaClouds.}.
%
%.
%
%Finally, several deployment modelling technologies may be related to OSTIA since their role is to model the deployment structure for Big data architectures such as in Celar\footnote{\url{https://github.com/CELAR/c-Eclipse}}, that is, a deployment modelling technology based on the TOSCA OASIS Standard\footnote{\url{https://www.oasis-open.org/apps/org/workgroup/tosca/}}. Celar may be used together with OSTIA In a scenario where OSTIA helps architecture refinement in function of infrastructure needs/requirements
