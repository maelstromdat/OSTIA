This section discusses some findings and the limitations of OSTIA.

\subsection{Findings and Continuous Architecting Insights}

OSTIA represents one humble, but significant step at supporting practically the necessities behind developing and maintaining high-quality big-data application architectures. In designing and developing OSTIA we encountered a number of insights that may aid continuous architecting.

First, we found (and observed in industrial practice) that it is often more useful to develop a quick-and-dirty but ``runnable" architecture topology than improving the topology at design time for a tentatively perfect execution. This is mostly the case with big-data applications that are developed stemming from previously existing topologies or applications. OSTIA hardcodes this way of thinking by supporting reverse-engineering and recovery of deployed topologies for their incremental improvement. Such improvement is helpful because these topologies running continuously on rented clusters and the refactoring can help in boosting the performance and therefore requiring less resources and less cost for the rented clusters. Although we did not carry out extensive qualitative or quantitative evaluation of OSTIA in this regard, we are planning additional industrial experiments for future work with the goal of increasing OSTIA usability and practical quality.

Second, big-data applications design is an extremely young and emerging field for which not many software design patterns have been discovered yet. The (anti-)patterns and approaches currently hardcoded into OSTIA are inherited from related fields, e.g., pattern- and cluster-based graph analysis. Nevertheless, OSTIA may also be used to investigate the existence of recurrent and effective design solutions (i.e., design patterns) for the benefit of big-data application design. We are improving OSTIA in this regard by experimenting on two fronts: (a) re-design and extend the facilities with which OSTIA supports anti-pattern detection; (b) run OSTIA on multiple big-data applications stemming from multiple technologies beyond Storm (e.g., Apache Spark, Hadoop Map Reduce, etc.) with the purpose of finding recurrent patterns. A similar approach may feature OSTIA as part of architecture trade-off analysis campaigns \cite{atam}.

Third, a step which is currently undersupported during big-data applications design is devising an efficient algorithmic breakdown of a workflow into an efficient topology. Conversely, OSTIA does support the linearisation and combination of multiple topologies, e.g., into a cascade. Cascading and similar super-structures may be an interesting investigation venue since they may reveal more efficient styles for big-data architectures beyond styles such as Lambda Architecture \cite{lambda} and Microservices \cite{balalaie2016microservices}. OSTIA may aid in this investigation by allowing the interactive and incremental improvement of multiple (combinations of) topologies together.

\subsection{Approach Limitations and Threats to Validity}\label{lim}

Although OSTIA shows promise both conceptually and as a practical tool, it shows several limitations.

First of all, OSTIA only supports only a limited set of DIA middleware technologies. Multiple other big-data frameworks such as Apache Spark, Samza, exist to support both streaming and batch processing. 

Second, OSTIA only allows to recover and evaluate previously-existing topologies, its usage is limited to design improvement and refactoring phases rather than design. Although this limitation may inhibit practitioners from using our technology, the (anti-)patterns and algorithmic approaches elaborated in this paper help designers and implementors to develop the reasonably good-quality and ``quick" topologies upon which to use OSTIA for continuous improvement.

Third, OSTIA does offer essential insights to aid deployment as well (e.g., separating or \emph{clustering} complex portions of a topology so that they may run on dedicated infrastructure) and therefore the tool may serve for the additional purpose of aiding deployment design. However, our tool was not designed to be used as a system that aids deployment planning and infrastructure design. Further research should be invested into combining on-the-fly technology such as OSTIA with more powerful solvers that determine infrastructure configuration details and similar technological tuning, e.g., the works by Peng et Al. \cite{PengGWRYC14} and similar.
%Rather, as specified previously in the introduction, OSTIA was meant to evaluate and increase the quality of topologies \emph{before} they enter into operation since the continuous improvement cycles connected to operating the topology and learning from operation are often costly and still greatly inefficient. \comment{but we provided evidences that we exploit the operation data for formal verification and performance improvements. this paragraph is a bit vague, please edit}

Fourth, although we were able to discover a number of recurrent anti-patterns to be applied during OSTIA analysis, we were not able to implement all of them in practice and in a manner which allows to spot both the anti-pattern and any problems connected with it. For example, detecting the ``Cycle-in topology" is already possible, however, OSTIA would not allow designers to understand the consequence of the anti-pattern, i.e., where in the infrastructure do the cycles cause troubles. Also, there are several features that are currently under implementation but not released within the OSTIA codebase, for example, the ``Persistent Data" and the ``Topology Cascading" features.

In the future we plan to tackle the above limitations furthering our understanding of streaming design as well as the support OSTIA offers to designers during continuous architecting.

