%% 
%% Copyright 2007, 2008, 2009 Elsevier Ltd
%% 
%% This file is part of the 'Elsarticle Bundle'.
%% ---------------------------------------------
%% 
%% It may be distributed under the conditions of the LaTeX Project Public
%% License, either version 1.2 of this license or (at your option) any
%% later version.  The latest version of this license is in
%%    http://www.latex-project.org/lppl.txt
%% and version 1.2 or later is part of all distributions of LaTeX
%% version 1999/12/01 or later.
%% 
%% The list of all files belonging to the 'Elsarticle Bundle' is
%% given in the file `manifest.txt'.
%% 

%% Template article for Elsevier's document class `elsarticle'
%% with numbered style bibliographic references
%% SP 2008/03/01

\documentclass[preprint,12pt]{elsarticle}

%% Use the option review to obtain double line spacing
%% \documentclass[authoryear,preprint,review,12pt]{elsarticle}

%% Use the options 1p,twocolumn; 3p; 3p,twocolumn; 5p; or 5p,twocolumn
%% for a journal layout:
%% \documentclass[final,1p,times]{elsarticle}
%% \documentclass[final,1p,times,twocolumn]{elsarticle}
%% \documentclass[final,3p,times]{elsarticle}
%% \documentclass[final,3p,times,twocolumn]{elsarticle}
%% \documentclass[final,5p,times]{elsarticle}
%% \documentclass[final,5p,times,twocolumn]{elsarticle}

%% For including figures, graphicx.sty has been loaded in
%% elsarticle.cls. If you prefer to use the old commands
%% please give \usepackage{epsfig}

%% The amssymb package provides various useful mathematical symbols
\usepackage{amssymb}

\usepackage{array}
\usepackage{graphicx}
\usepackage{float}
\usepackage{supertabular}
\usepackage[table]{xcolor}



% Some very useful LaTeX packages include:
% (uncomment the ones you want to load)
\providecommand{\tabularnewline}{\\}
\usepackage{booktabs, multicol, multirow}
\usepackage[acronym,nomain]{glossaries}
\usepackage[inline]{enumitem}
\usepackage{algorithm}
\usepackage{algorithmicx}
\usepackage{algpseudocode}
\usepackage{amsmath}
\usepackage{amsfonts}
\usepackage{subfigure}
\usepackage{grffile}
\usepackage{tabularx}
\usepackage{colortbl}
\usepackage{hhline}
\usepackage{hyperref}
\usepackage{here}
\usepackage{graphicx}
% % % % % % % % % % % % % % % % % % % % % % % % % % % % % % % % % % % %


% % % % % % % % % % % % % % % % % % % % % % % % % % % % % % % % % % % % new commands % % %
\usepackage[colorinlistoftodos,prependcaption,textsize=tiny]{todonotes}
%
% Convenience commands for the paper editing process
%
\newcommand{\comment}[1]{{\textbf{\color{red}[#1]}}}
\newcommand{\fixed}[1]{{\textbf{\color{blue}[#1]}}}

%
% Convenience commands for references
%
\newcommand{\secref}[1]{Section~\ref{sec:#1}}
\newcommand{\tabref}[1]{Table~\ref{tab:#1}}
\newcommand{\figref}[1]{Figure~\ref{fig:#1}}
\newcommand{\quotes}[1]{``#1''}

\newcommand{\argmin}{\arg\!\min}
\newcommand{\argmax}{\arg\!\max}


\newtheorem{definition}{Definition}

\renewcommand{\algorithmicrequire}{\textbf{Input:}}
\renewcommand{\algorithmicensure}{\textbf{Output:}}

% % % % % % % % % % % % % % % % % % % % % % % % % % % % % % % % % % % %

% % % % % % % % % % % % % % % % % % % % % % % % % % % % % % % % % % % %
% Acronym definitions
\newacronym{sps}{SPS}{Stream Processing Systems}
\newacronym{pe}{PE}{Processing Elements}
\newacronym{dag}{DAG}{Directed Acyclic Graph}

% % % % % % % % % % % % % % % % % % % % % % % % % % % % % % % % % % % %


% Some very useful LaTeX packages include:
% (uncomment the ones you want to load)


% *** MISC UTILITY PACKAGES ***
%
%\usepackage{ifpdf}
% Heiko Oberdiek's ifpdf.sty is very useful if you need conditional
% compilation based on whether the output is pdf or dvi.
% usage:
% \ifpdf
%   % pdf code
% \else
%   % dvi code
% \fi
% The latest version of ifpdf.sty can be obtained from:
% http://www.ctan.org/pkg/ifpdf
% Also, note that IEEEtran.cls V1.7 and later provides a builtin
% \ifCLASSINFOpdf conditional that works the same way.
% When switching from latex to pdflatex and vice-versa, the compiler may
% have to be run twice to clear warning/error messages.






% *** CITATION PACKAGES ***
%
%\usepackage{cite}
% cite.sty was written by Donald Arseneau
% V1.6 and later of IEEEtran pre-defines the format of the cite.sty package
% \cite{} output to follow that of the IEEE. Loading the cite package will
% result in citation numbers being automatically sorted and properly
% "compressed/ranged". e.g., [1], [9], [2], [7], [5], [6] without using
% cite.sty will become [1], [2], [5]--[7], [9] using cite.sty. cite.sty's
% \cite will automatically add leading space, if needed. Use cite.sty's
% noadjust option (cite.sty V3.8 and later) if you want to turn this off
% such as if a citation ever needs to be enclosed in parenthesis.
% cite.sty is already installed on most LaTeX systems. Be sure and use
% version 5.0 (2009-03-20) and later if using hyperref.sty.
% The latest version can be obtained at:
% http://www.ctan.org/pkg/cite
% The documentation is contained in the cite.sty file itself.






% *** GRAPHICS RELATED PACKAGES ***
%


  % declare the path(s) where your graphic files are
  % \graphicspath{{../pdf/}{../jpeg/}}
  % and their extensions so you won't have to specify these with
  % every instance of \includegraphics
  % \DeclareGraphicsExtensions{.pdf,.jpeg,.png}

  % or other class option (dvipsone, dvipdf, if not using dvips). graphicx
  % will default to the driver specified in the system graphics.cfg if no
  % driver is specified.
  % \usepackage[dvips]{graphicx}
  % declare the path(s) where your graphic files are
  % \graphicspath{{../eps/}}
  % and their extensions so you won't have to specify these with
  % every instance of \includegraphics
  % \DeclareGraphicsExtensions{.eps}

% graphicx was written by David Carlisle and Sebastian Rahtz. It is
% required if you want graphics, photos, etc. graphicx.sty is already
% installed on most LaTeX systems. The latest version and documentation
% can be obtained at:
% http://www.ctan.org/pkg/graphicx
% Another good source of documentation is "Using Imported Graphics in
% LaTeX2e" by Keith Reckdahl which can be found at:
% http://www.ctan.org/pkg/epslatex
%
% latex, and pdflatex in dvi mode, support graphics in encapsulated
% postscript (.eps) format. pdflatex in pdf mode supports graphics
% in .pdf, .jpeg, .png and .mps (metapost) formats. Users should ensure
% that all non-photo figures use a vector format (.eps, .pdf, .mps) and
% not a bitmapped formats (.jpeg, .png). The IEEE frowns on bitmapped formats
% which can result in "jaggedy"/blurry rendering of lines and letters as
% well as large increases in file sizes.
%
% You can find documentation about the pdfTeX application at:
% http://www.tug.org/applications/pdftex


\usepackage{pgf}
\usepackage{tikz}
\usetikzlibrary{arrows,automata}
\usetikzlibrary{shapes.misc}

\tikzset{cross/.style={cross out, draw=black, minimum size=3*(#1-\pgflinewidth), inner sep=0pt, outer sep=0pt},
	%default radius will be 1pt. 
	cross/.default={10pt}}



% *** MATH PACKAGES ***
%
%\usepackage{amsmath}
% A popular package from the American Mathematical Society that provides
% many useful and powerful commands for dealing with mathematics.
%
% Note that the amsmath package sets \interdisplaylinepenalty to 10000
% thus preventing page breaks from occurring within multiline equations. Use:
%\interdisplaylinepenalty=2500
% after loading amsmath to restore such page breaks as IEEEtran.cls normally
% does. amsmath.sty is already installed on most LaTeX systems. The latest
% version and documentation can be obtained at:
% http://www.ctan.org/pkg/amsmath





% *** SPECIALIZED LIST PACKAGES ***
%

% algorithmic.sty was written by Peter Williams and Rogerio Brito.
% This package provides an algorithmic environment fo describing algorithms.
% You can use the algorithmic environment in-text or within a figure
% environment to provide for a floating algorithm. Do NOT use the algorithm
% floating environment provided by algorithm.sty (by the same authors) or
% algorithm2e.sty (by Christophe Fiorio) as the IEEE does not use dedicated
% algorithm float types and packages that provide these will not provide
% correct IEEE style captions. The latest version and documentation of
% algorithmic.sty can be obtained at:
% http://www.ctan.org/pkg/algorithms
% Also of interest may be the (relatively newer and more customizable)
% algorithmicx.sty package by Szasz Janos:
% http://www.ctan.org/pkg/algorithmicx




% *** ALIGNMENT PACKAGES ***
%
\usepackage{array}
% Frank Mittelbach's and David Carlisle's array.sty patches and improves
% the standard LaTeX2e array and tabular environments to provide better
% appearance and additional user controls. As the default LaTeX2e table
% generation code is lacking to the point of almost being broken with
% respect to the quality of the end results, all users are strongly
% advised to use an enhanced (at the very least that provided by array.sty)
% set of table tools. array.sty is already installed on most systems. The
% latest version and documentation can be obtained at:
% http://www.ctan.org/pkg/array


% IEEEtran contains the IEEEeqnarray family of commands that can be used to
% generate multiline equations as well as matrices, tables, etc., of high
% quality.




% *** SUBFIGURE PACKAGES ***
%\ifCLASSOPTIONcompsoc
%  \usepackage[caption=false,font=normalsize,labelfont=sf,textfont=sf]{subfig}
%\else
%  \usepackage[caption=false,font=footnotesize]{subfig}
%\fi
% subfig.sty, written by Steven Douglas Cochran, is the modern replacement
% for subfigure.sty, the latter of which is no longer maintained and is
% incompatible with some LaTeX packages including fixltx2e. However,
% subfig.sty requires and automatically loads Axel Sommerfeldt's caption.sty
% which will override IEEEtran.cls' handling of captions and this will result
% in non-IEEE style figure/table captions. To prevent this problem, be sure
% and invoke subfig.sty's "caption=false" package option (available since
% subfig.sty version 1.3, 2005/06/28) as this is will preserve IEEEtran.cls
% handling of captions.
% Note that the Computer Society format requires a larger sans serif font
% than the serif footnote size font used in traditional IEEE formatting
% and thus the need to invoke different subfig.sty package options depending
% on whether compsoc mode has been enabled.
%
% The latest version and documentation of subfig.sty can be obtained at:
% http://www.ctan.org/pkg/subfig




% *** FLOAT PACKAGES ***
%
%\usepackage{fixltx2e}
% fixltx2e, the successor to the earlier fix2col.sty, was written by
% Frank Mittelbach and David Carlisle. This package corrects a few problems
% in the LaTeX2e kernel, the most notable of which is that in current
% LaTeX2e releases, the ordering of single and double column floats is not
% guaranteed to be preserved. Thus, an unpatched LaTeX2e can allow a
% single column figure to be placed prior to an earlier double column
% figure.
% Be aware that LaTeX2e kernels dated 2015 and later have fixltx2e.sty's
% corrections already built into the system in which case a warning will
% be issued if an attempt is made to load fixltx2e.sty as it is no longer
% needed.
% The latest version and documentation can be found at:
% http://www.ctan.org/pkg/fixltx2e


%\usepackage{stfloats}
% stfloats.sty was written by Sigitas Tolusis. This package gives LaTeX2e
% the ability to do double column floats at the bottom of the page as well
% as the top. (e.g., "\begin{figure*}[!b]" is not normally possible in
% LaTeX2e). It also provides a command:
%\fnbelowfloat
% to enable the placement of footnotes below bottom floats (the standard
% LaTeX2e kernel puts them above bottom floats). This is an invasive package
% which rewrites many portions of the LaTeX2e float routines. It may not work
% with other packages that modify the LaTeX2e float routines. The latest
% version and documentation can be obtained at:
% http://www.ctan.org/pkg/stfloats
% Do not use the stfloats baselinefloat ability as the IEEE does not allow
% \baselineskip to stretch. Authors submitting work to the IEEE should note
% that the IEEE rarely uses double column equations and that authors should try
% to avoid such use. Do not be tempted to use the cuted.sty or midfloat.sty
% packages (also by Sigitas Tolusis) as the IEEE does not format its papers in
% such ways.
% Do not attempt to use stfloats with fixltx2e as they are incompatible.
% Instead, use Morten Hogholm'a dblfloatfix which combines the features
% of both fixltx2e and stfloats:
%
% \usepackage{dblfloatfix}
% The latest version can be found at:
% http://www.ctan.org/pkg/dblfloatfix




% *** PDF, URL AND HYPERLINK PACKAGES ***
%
% *** MISC UTILITY PACKAGES ***
%
%\usepackage{ifpdf}
% Heiko Oberdiek's ifpdf.sty is very useful if you need conditional
% compilation based on whether the output is pdf or dvi.
% usage:
% \ifpdf
%   % pdf code
% \else
%   % dvi code
% \fi
% The latest version of ifpdf.sty can be obtained from:
% http://www.ctan.org/tex-archive/macros/latex/contrib/oberdiek/
% Also, note that IEEEtran.cls V1.7 and later provides a builtin
% \ifCLASSINFOpdf conditional that works the same way.
% When switching from latex to pdflatex and vice-versa, the compiler may
% have to be run twice to clear warning/error messages.

\usepackage{balance}
\usepackage{url}
\usepackage{rotating}


% *** CITATION PACKAGES ***
%
%\usepackage{cite}
% cite.sty was written by Donald Arseneau
% V1.6 and later of IEEEtran pre-defines the format of the cite.sty package
% \cite{} output to follow that of IEEE. Loading the cite package will
% result in citation numbers being automatically sorted and properly
% compressed/ranged. e.g., [1], [9], [2], [7], [5], [6] without using
% cite.sty will become [1], [2], [5]--[7], [9] using cite.sty. cite.sty's
% \cite will automatically add leading space, if needed. Use cite.sty's
% noadjust option (cite.sty V3.8 and later) if you want to turn this off.
% cite.sty is already installed on most LaTeX systems. Be sure and use
% version 4.0 (2003-05-27) and later if using hyperref.sty. cite.sty does
% not currently provide for hyperlinked citations.
% The latest version can be obtained at:
% http://www.ctan.org/tex-archive/macros/latex/contrib/cite/
% The documentation is contained in the cite.sty file itself.



%% The amsthm package provides extended theorem environments
%% \usepackage{amsthm}

%% The lineno packages adds line numbers. Start line numbering with
%% \begin{linenumbers}, end it with \end{linenumbers}. Or switch it on
%% for the whole article with \linenumbers.
%% \usepackage{lineno}

\journal{Journal of Systems and Software}

\begin{document}

\begin{frontmatter}

%% Title, authors and addresses

%% use the tnoteref command within \title for footnotes;
%% use the tnotetext command for theassociated footnote;
%% use the fnref command within \author or \address for footnotes;
%% use the fntext command for theassociated footnote;
%% use the corref command within \author for corresponding author footnotes;
%% use the cortext command for theassociated footnote;
%% use the ead command for the email address,
%% and the form \ead[url] for the home page:
%% \title{Title\tnoteref{label1}}
%% \tnotetext[label1]{}
%% \author{Name\corref{cor1}\fnref{label2}}
%% \ead{email address}
%% \ead[url]{home page}
%% \fntext[label2]{}
%% \cortext[cor1]{}
%% \address{Address\fnref{label3}}
%% \fntext[label3]{}
\title{Continuous Architecting of Stream-Based Systems}
%% use optional labels to link authors explicitly to addresses:
%% \author[label1,label2]{}
%% \address[label1]{}
%% \address[label2]{}

\author[1]{Marcello Bersani, Francesco Marconi, Damian A. Tamburri}
\ead{$[$marcellomaria.bersani, francesco.marconi, damianandrew.tamburri$]$@polimi.it}
\author[2]{\\Andrea Nodari, Pooyan Jamshidi}
\ead{$[$p.jamshidi,a.nodari15$]$@imperial.co.uk}
\address[1]{Politecnico di Milano, Milan, Italy}
\address[2]{Imperial College London, UK}

\begin{abstract}
%% Text of abstract
Big data architectures have been gaining momentum in recent years. For
instance, Twitter uses stream processing frameworks like Storm to analyse billions of tweets per minute and learn the trending topics. However, architectures that process big data involve many different components interconnected via semantically different connectors making it a difficult task for software architects to refactor the initial designs. As an aid to designers and developers, we developed OSTIA (On-the-fly Static Topology Inference Analysis) that allows: (a) visualising big data architectures for the purpose of design-time refactoring while maintaining constraints that would only be evaluated at later stages such as deployment and run-time; (b) detecting the occurrence of common anti-patterns across big data architectures; (c) exploiting software verification techniques on the elicited architectural models. This paper illustrates OSTIA and evaluates its uses and benefits on three industrial-scale case studies.
\end{abstract}

%\begin{keyword}
%%% keywords here, in the form: keyword \sep keyword
%social debt \sep technical debt \sep social software engineering \sep software communities \sep organisational structure \sep social structure 
%%% PACS codes here, in the form: \PACS code \sep code
%
%%% MSC codes here, in the form: \MSC code \sep code
%%% or \MSC[2008] code \sep code (2000 is the default)
%
%\end{keyword}
\end{frontmatter}

%% \linenumbers

%% main text


\section{Introduction}
\label{intro}
%%\begin{itemize}
%%\item I would follow the path of the abstract, we should probably provide some numbers and info on storm
%%\item mind you we should stress on the innovative aspects of the paper and tech. there is nothing strictly related to it
%%\item we should comment on what could be done with OSTIA in combination with Eclipse Based tech.
%%\end{itemize}
%%%Big data architectures have been gaining momentum in the last few years. For example, Twitter uses complex Stream topologies featuring frameworks like Storm to analyse and learn trending topics from billions of tweets per minute. However, verifying the consistency of said topologies often requires de- ployment on multi-node clusters and can be expensive as well as time consuming. As an aid to designers and developers evaluating their Stream topologies at design-time, we developed OSTIA, that is, ?On-the-fly Storm Topology Inference Analysis?. OSTIA allows reverse-engineering of Storm topologies so that designers and developers may: (a) use previously existing model- driven verification&validation techniques on elicited models; (b) visualise and evaluate elicited models against consistency checks that would only be available at deployment and run-time. We illustrate the uses and benefits of OSTIA on three real-life industrial case studies.
%%%%%%
%%%%%% intro needs a bit of refinement with what we say in the title and a few more definitions should be included (e.g., about streaming and what it represents or why topologies ?are? the Big Data architecture)? Perhaps we should also increase the stress and focus on Quality and deployability aspects (i.e., the main topics of next year?s QoSA) in the intro, and how OSTIA aids at improving these aspects

Big data applications process large amounts of data for the purpose of gaining key business intelligence through complex analytics such as machine-learning \cite{bdsurvey, ml4bd}. These applications are receiving increased attention in the last years given their ability to yield competitive advantage by direct investigation of user needs and trends hidden in the enormous quantities of data produced daily by the average Internet user. According to Gartner\footnote{\url{http://www.gartner.com/newsroom/id/2637615}} business intelligence and analytics applications will remain a top focus for CIOs until at least 2017-2018.
However, the cost of ownership of the systems that process big data analytics are high due to high infrastructure costs, steep learning curves for the different frameworks involved in designing and developing the applications, such as Apache Storm\footnote{\url{http://storm.apache.org/}}, Apache Spark\footnote{\url{http://spark.apache.org/}} or Apache Hadoop\footnote{\url{https://hadoop.apache.org/}} and complexities in large-scale architectures.

In our own experience with designing and developing for big data, we observed that a key complexity lies in quickly and continuously evaluating the effectiveness of big-data architectures. Effectiveness, in big data terms, means being able to design, deploy, operate, refactor and then (re-)deploy architectures continuously and consistently with runtime restrictions of imposed by the development frameworks. Storm, for example, requires the processing elements to represent a Directed-Acyclic-Graph (DAG). In toy topologies, such constraints can be effectively checked manually, however, when the number of nodes in such architectures increases to real-life industrial scale architecture, it is enormously difficult to verify whether the structural DAG constraints.
%\textbf{TODO: can we add an example of said consistency checks/issues?} \\
We argue that this effectiveness can be maintained starting from design time, by enacting a continuous architecting of big data architectures consistently with a DevOps organisational structure \cite{ossslr,devops}. Such structure eases the (re-)deployability of big data architectures and saves the effort to run trial-and-error experiments on expensive infrastructure.

To sustain this argument, we developed OSTIA, that stands for: ``On-the-fly Static Topology Inference Analysis". OSTIA allows designers and developers to infer the application architecture through on-the-fly reverse-engineering and architecture recovery \cite{archrec}. During this inference step, OSTIA analyses the architecture to verify whether it is consistent with restrictions and constraints of the underlying development frameworks (e.g., DAG constraints). In an effort to tackle said complexities in a DevOps fashion, OSTIA is engineered to act as a mechanism that closes the feedback loop between operational data architectures (Ops phase) and their refactoring phase (Dev phase).

Currently, OSTIA focuses on Apache Storm, i.e., one of the most famous and established real-time stream processing engine \cite{storm, toshniwal2014storm}. The core element of Storm, is called \emph{topology}, which represents the architecture of the processing components of the application (from now we use topology and architecture interchangeably).

OSTIA hardcodes intimate knowledge on the streaming development framework (Storm, in our case) and its dependence structure in the form of a meta-model \cite{mda}. This knowledge is necessary to make sure that elicited topologies are correct, so that models may be used in at least four scenarios: (a) realising an exportable visual representation of the developed topologies; (b) verifying structural constraints on topologies that would only become evident during infrastructure setup or runtime operation; (c) verifying the topologies against anti-patterns \cite{patternoriented2000} that may lower performance and limit deployability/executability; (d) finally, use topologies for further analysis, e.g., through model verification \cite{icsoft}.
%\comment{PJ: please list the 3 topologies here we claim there are three but we name only 1 here}. 

This paper outlines OSTIA, elaborating major usage scenarios, benefits and limitations. Also, we evaluate OSTIA using case-study research and formal validation to conclude that OSTIA does in fact provide valuable insights for continuous architecting.

The rest of the paper is structured as follows. Section \ref{ra} outlines our research design. Sections \ref{rs}, \ref{sec:anti-pattern} and \ref{algo} describe OSTIA, discussing the (anti-)patterns it supports and its usage scenarios . Section \ref{eval} evaluates OSTIA while Section \ref{disc} discusses results and evaluation outlining OSTIA limitations and threats to validity. Finally, Sections \ref{rw} and \ref{conc} report related work and conclude the paper.



\section{Research Design}
\label{ra}
%\begin{itemize}
%\item so we had a focus group to actually elaborate the aprroach
%\item then we used explorative prototyping to elicit the initial version of the prototype and then refined that by means of case study, we could mention that we used ATC as an action-research source (this is what we are doing now internally in WP2 actually)
%\item ...
%\end{itemize}
%
%The work we elaborated in this paper is stemming from the following research question:
%
%\begin{center}
%\emph{``What are the sub-optimal structural"}
%\end{center}
%% \comment{I would say if we connect it to the special need in DICE would makes more sense?}
%
%This research question emerged as part of our work within the DICE EU H2020 project~\cite{dice2020}
%%\footnote{\url{http://www.dice-h2020.eu/}} 
%where we evaluated our case-study owners' scenarios and found that their continuous architecting needs were: (a) focusing on the topological abstractions and surrounding architectural specifications; (b) focusing on bridging the gap between insights from Ops to (re-)work at the Dev level; (c) their needs primarily consisted in maintaining architectural consistency during refactoring.
%In pursuit of the research question above, 

From a methodological perspective, the results outlined in this paper were elaborated as follows and made concrete through the actions in Sec.~\ref{sec:antipatternextraction} and \ref{sec:researchsolutioneval}.

\subsection{Extracting Anti-Patterns for Big Data Applications}\label{sec:antipatternextraction}

The anti-patterns illustrated in this paper were initially elaborated within 3 structured focus-groups \cite{focusgroup} involving practitioners from a different organization in each focus-group round; subsequently, we interviewed 2 domain-expert (5+ years of experience) researchers on big data technologies as a control group. The data was analyzed with a simple card-sorting exercise and cross-referenced the patterns thus emerged with the ones emerging from our interview-based control group; \todoMB{}{Dam, non capsico la frase prima del punto-e-virgola} disagreement between the two groups was evaluated Inter-Rater Reliability assessment using the well-known Krippendorf Alpha coefficient \cite{content} (assessment of $K_{alpha}=0,89$). 

Table \ref{tabba} outlines the population we used for this part of the study. The practitioners were simply required to elaborate on the most frequent structural and anti-patterns they encountered on their DIA design and experimentation. 

\begin{table}
\caption{Focus-Groups population outline.}\label{tabba}
\begin{tabular}{|c|c|c|c|}
\hline 
\textbf{Role} & \textbf{\#Participants} & \textbf{Mean Age} & \textbf{Mean Exp. With DIAs (\#months)}\tabularnewline
\hline 
Architect & 3 & 35,3 & 17,3\tabularnewline
\hline 
Developer & 4 & 27,7 & 36,2\tabularnewline
\hline 
Operator & 5 & 31,1 & 38,1\tabularnewline
\hline 
Manager & 3 & 44,2 & 18,4\tabularnewline
\hline 
\end{tabular}
\end{table}

The focus-group sessions were structured as follows: (a) the practitioners were presented with a data-intensive architectural design using standard UML structure and behavior representations (a component view and an activity view \cite{NittoJGST16}); (b) the practitioners were asked to identify and discuss any bottlenecks or structural limitations in the outlined designs; (c) finally, the practitioners were asked to illustrate any other anti-pattern the showcased topologies did not contain.
%\todoMB{}{Spiegare meglio il focus-group}.
%Following the focus-group, through self-ethnography \cite{selfeth} and brainstorming we identified the series of essential consistency checks, algorithmic evaluations as well as anti-patterns that can now be applied through OSTIA while recovering an architectural representation for Storm topologies. 

%We designed OSTIA
%%\footnote{\url{https://github.com/maelstromdat/OSTIA}} 
%to support the incremental and iterative refinement of streaming topologies based on the incremental discovery and correction of the anti-patterns.
%
%\begin{figure*}
%  \centering
%  \includegraphics[width=12cm]{images/socialsensorother}
%  \caption{A sample Storm topology (readable from left to right) using an UML object diagram in the SocialSensor App, notes identify types for nodes (i.e., Bolts or Spouts).}
%  \label{socialsensor-topology}
%\end{figure*}

\subsection{Research Solution Evaluation}\label{sec:researchsolutioneval}

OSTIA's evaluation is threefold. 

First, we evaluated our solution using an industrial case-study offered by one of the industrial partners in the DICE EU H2020 Project consortium~\cite{dice2020}.
%\footnote{\url{http://www.dice-h2020.eu/}}.  
The
partner in question uses open-source social-sensing software to elaborate a
subscription-based big-data application that: (a) aggregates news assets from
various sources (e.g., Twitter, Facebook, etc.) based on user-desired
specifications (e.g., topic, sentiment, etc.); (b) presents and allows the
manipulation of data. The application in question is based on the SocialSensor
App~\cite{socialsensor}
%\footnote{\url{https://github.com/socialsensor}} 
%(see
%Fig. \ref{socialsensor-topology} for a sample realised using a simple UML object diagram).  In particular, the topology in
%Fig. \ref{socialsensor-topology} extracts data from sources and divides and arranges contents based on type (e.g., article vs. media),
%later updating a series of databases (e.g., Redis) with these elaborations.
%
which features the combined
action of three complex streaming topologies based on Apache Storm. The
models that OSTIA elicited from this application were showcased to our
industrial partner in a focus group aimed at establishing the value of insights
produced as part of OSTIA-based analyses. Our qualitative assessment was based
on questionnaires and open discussion.

Second, to further confirm the validity of OSTIA analyses and support, we
applied it on two open-source applications featuring Big-Data analytics, namely: (a) the DigitalPebble application,
%\footnote{\url{https://github.com/DigitalPebble/storm-crawler}}, 
``A text classification API in Java originally developed by DigitalPebble Ltd.
The API is independent from the ML implementations and can be used as a front
end to various ML algorithms''~\cite{storm-crawler}; (b) the StormCV application, 
%\footnote{\url{https://github.com/sensorstorm/StormCV}} 
``StormCV
enables the use of Apache Storm for video processing by adding computer vision
(CV) specific operations and data model; the platform enables the development of
distributed video processing pipelines which can be deployed on Storm clusters"~\cite{stormCV}.

%\begin{itemize}
%\item elaborate on the case-study partner
%\item elaborate on the case at hand for that partner
%\item elaborate on the origin of the application and how it uses social-sensor
%\item also, elaborate on the case by NETF which starts and stems from KILLRWEATHER
%\item should we elaborate on something else?
%\end{itemize}
%\textbf{TODO: @anyone, feel free to elaborate more!!}
Third, finally, as part of the OSTIA extension recapped in this manuscript, we applied formal verification approaches using the
Zot~\cite{zot}
%\footnote{\url{https://github.com/fm-polimi/zot}} 
model-checker following an approach tailored from previous work \cite{icsoft,BRS15}.

%\comment{this section need a bit of refactoring it's not focused enough}



\section{Research Solution}
\label{rs}
\textbf{@Pooyan,Andrea: here we should probably elaborate on OSTIA's architecture and the design principles that led us to define it as such... also we might want to elaborate on its components, the structure I'm suggesting below is merely tentative but it will give us ahead start!!}

\begin{itemize}
\item OSTIA Architecture 
\item we should probably elaborate the architecture part (or on a separate "implementation" part or paragraph) with a link to the downloadable technology - @Andrea: can we bundle it up as plugin for Eclipse? E.g., somehow using RCP?
\item OSTIA Antipatterns Module
\item OSTIA Visualisation Module 
\item OSTIA extensibility
\item OSTIA explanation of use and simple usage scenario
\end{itemize}
\section{Topology Design Anti-Patterns}

This section and section \ref{algo} elaborate on the Anti-patterns and algorithmic analysis supported in OSTIA. All figures in these sections use a simple graph-like notation where nodes may be any topological element (e.g., Spouts or Bolts in Apache Storm terms) while edges are to be interpreted as directed data-flow connections.

\label{sec:anti-pattern}
This elaborates on the anti-patterns we elicited through self-ethnography. These anti-patterns are elaborated further within OSTIA to allow for their detection during streaming topology inference analysis.

\subsection{Multi-Anchoring}
In order to guarantee fault-tolerant stream processing, tuples processed by bolts needs to be anchored with the unique id of the bolt and be passed to multiple acknowledgers (or ``ackers" in short) in the topology. In this way, ackers can keep track of tuples in the topology.\\
%\emph{\bf TODO: what is the consequence of these anti-patterns? How does OSTIA detect?}
%\emph{\bf Multi-anchoring is not supported at the moment. Besides, I am not sure it is an anti-patter but rather a design decision}

\begin{figure}[H]
	\begin{center}
		\includegraphics[width=2.5cm]{images/multi-anchoring}
		\caption{Multi-anchoring.}
		\label{fig:multi-anchoring}
	\end{center}
\end{figure}

\subsection{Cycle-in Topology}

Technically, it is possible to have cycle in Storm topologies. An infinite cycle of processing would create an infinite tuple tree and make it impossible for Storm to ever acknowledge spout emitted tuples. Therefore, cycles should be avoided or resulting tuple trees should be investigated additionally to make sure they terminate at some point and under a specified series of conditions. The anti-pattern itself may lead to infrastructure overloading and therefore increased costs.
%\emph{\bf A topology is already an infinite stream of tuple, the problem could be the overloading of some machines}
%\emph{\bf At the cycle-detection is not supported (even if it is easy to implement)}

\begin{figure}[H]
	\begin{center}
		\includegraphics[width=2cm]{images/cycle}
		\caption{Cycle-in Topology.}
		\label{fig:cycle}
	\end{center}
\end{figure}

\subsection{Persistent Data}

This pattern covers the circumstance wherefore if two processing elements need to update a same entity in a storage, there should be a consistency mechanism in place. OSTIA offers limited support to this feature, which we plan to look into more carefully for future work. More details on this support are discussed in the approach limitations section (see Sec. \ref{lim}).
%\emph{\bf Ostia does not support this. BTW is this static analysis? if not, is it off-topic?}

\begin{figure}[H]
	\begin{center}
		\includegraphics[width=4cm]{images/persistence}
		\caption{Concurrency management in case of Persistent Data circumstances.}
		\label{fig:persistence}
	\end{center}
\end{figure}


\subsection{Computation funnel}
A computational funnel emerges when there is not a path from data source (spout) to the bolts that sends out the tuples off the topology to another topology through a messaging framework or through a storage. This circumstance should be dealt with since it may compromise the availability of results under the desired performance restrictions.

\begin{figure}[H]
	\begin{center}
		\includegraphics[width=5cm]{images/funnel}
		\caption{computation funnel.}
		\label{fig:funnel}
	\end{center}
\end{figure}

\section{Algorithmic Analysis on Stream Topologies}\label{algo}

\subsection{fan-in/fan-out}

For each element of the topology, fan-in is the number of incoming
streams. Conversely, fan-out is the number outgoing streams. In the case of
bolts, both in and out streams are internal to the topology. For Spouts,
incoming streams are the data sources of the topology (e.g., message brokers,
APIs, etc).

\begin{figure}[H]
	\begin{center}
		\includegraphics[width=3cm]{images/fan-in-out}
		\caption{Fan-in fan-out in Stream topologies.}
		\label{fig:fan}
	\end{center}
\end{figure}

This algorithmic manipulation allows to visualise instances where fan-in and fan-out numbers are differing.

\subsection{topology cascading}

By topology cascading, we mean connecting two different Storm topologies via a messaging framework (e.g., Apache Kafka). This circumstance, which is actually part of our evaluation and case-studies, may raise the complexity of the overarching topology to unacceptable levels and may require additional attention. OSTIA support for this feature is still limited, more details on this and similar limitations are discussed in Section \ref{lim}.
%\emph{\bf Ostia does not support this. I can't think of an easy way to implement it}

\begin{figure}[H]
	\begin{center}
		\includegraphics[width=6cm]{images/cascading}
		\caption{cascading.}
		\label{fig:cascading}
	\end{center}
\end{figure}

This algorithmic manipulation allows to combine multiple cascading topologies.

\subsection{Topology clustering}
Identifying the coupled processing elements and put the in a cluster in a way that elements in a cluster have high cohesion and less coupled with the elements in other clusters. Simple clustering or Social-Network Analysis mechanisms can be used to infer clusters. These clusters may require additional attention since they could turn out to become bottlenecks. Reasoning more deeply on clusters and their resolution may lead to establishing the Storm scheduling policy best-fitting with the application at hand.
%\emph{\bf Does it relates with Storm scheduling?}

\begin{figure}[H]
	\begin{center}
		\includegraphics[width=5cm]{images/clustering}
		\caption{clustering.}
		\label{fig:clustering}
	\end{center}
\end{figure}

\subsection{Linearising a topology}

Sorting the processing elements in a topology in a way that topology looks more linear, visually. This step ensures that visual investigation and evaluation of the structural complexity of the topology is possible by direct observation. It is sometimes essential to provide such a visualisation to evaluate how to refactor the topology based on emerging needs.

\begin{figure}[H]
	\begin{center}
		\includegraphics[width=5cm]{images/linearizing}
		\caption{linearizing.}
		\label{fig:linearizing}
	\end{center}
\end{figure}


\section{Evaluation}
\label{eval}
\textbf{@Marcello,Francesco: we should also probably elaborate on the kind of verification technique we are using and how that can help in evaluating the topology.. remember here we do not have the DICE restriction so we can mention any kind of analysis that it would be possible to run, also analyses that are currently in the hands of other DICE partners!!}

\begin{itemize}
\item we can use the ATC case study as much as we want - that yields already three topologies that we can infer
\item ATC has agreed that we can mention their role in this exercise, I also showed them the topology that we elicited basically with OSTIA and they already made considerations on how to improve it
\item in the evaluation we should also comment on how OSTIA can help you in visualizing the application topology that you may be considering to use by reusing a big-data application for something else... visualising the application topology and analysing it may allow you to improve it while you are using it as a starting point for your application
\item another application that we can use is the one that NETF is considering for their own scenario, KILLRWEATHER - \url{https://github.com/killrweather/killrweather}
\item any additional case that we can run?
\item what do the results show? do we have a way to quickly quantify the time that is saved by using this approach? e.g., the time that is saved in setting up and running the infrastructure and how much would that time saved have costed these could be valuable evaluation insights
\end{itemize}

\section{Discussion}
\label{disc}
This section discusses some findings and the limitations of OSTIA.

\subsection{Findings and Continuous Architecting Insights}

OSTIA represents one humble, but significant step at supporting practically the necessities behind developing and maintaining high-quality big-data application architectures. In designing and developing OSTIA we encountered a number of insights that may aid continuous architecting.

First, we found (and observed in industrial practice) that it is often more useful to develop a quick-and-dirty but ``runnable" architecture topology than improving the topology at design time for a tentatively perfect execution. This is mostly the case with big-data applications that are developed stemming from previously existing topologies or applications. OSTIA hardcodes this way of thinking by supporting reverse-engineering and recovery of deployed topologies for their incremental improvement. Such improvement is helpful because these topologies running continuously on rented clusters and the refactoring can help in boosting the performance and therefore requiring less resources and less cost for the rented clusters. Although we did not carry out extensive qualitative or quantitative evaluation of OSTIA in this regard, we are planning additional industrial experiments for future work with the goal of increasing OSTIA usability and practical quality.

Second, big-data applications design is an extremely young and emerging field for which not many software design patterns have been discovered yet. The (anti-)patterns and approaches currently hardcoded into OSTIA are inherited from related fields, e.g., pattern- and cluster-based graph analysis. Nevertheless, OSTIA may also be used to investigate the existence of recurrent and effective design solutions (i.e., design patterns) for the benefit of big-data application design. We are improving OSTIA in this regard by experimenting on two fronts: (a) re-design and extend the facilities with which OSTIA supports anti-pattern detection; (b) run OSTIA on multiple big-data applications stemming from multiple technologies beyond Storm (e.g., Apache Spark, Hadoop Map Reduce, etc.) with the purpose of finding recurrent patterns. A similar approach may feature OSTIA as part of architecture trade-off analysis campaigns \cite{atam}.

Third, a step which is currently undersupported during big-data applications design is devising an efficient algorithmic breakdown of a workflow into an efficient topology. Conversely, OSTIA does support the linearisation and combination of multiple topologies, e.g., into a cascade. Cascading and similar super-structures may be an interesting investigation venue since they may reveal more efficient styles for big-data architectures beyond styles such as Lambda Architecture \cite{lambda} and Microservices \cite{balalaie2016microservices}. OSTIA may aid in this investigation by allowing the interactive and incremental improvement of multiple (combinations of) topologies together.

\subsection{Approach Limitations and Threats to Validity}\label{lim}

Although OSTIA shows promise both conceptually and as a practical tool, it shows several limitations.

First of all, OSTIA only supports streaming topologies enacted using Storm. Multiple other big-data frameworks such as Apache Spark, Samza exist to support both streaming and batch processing. 

Second, OSTIA only allows to recover and evaluate previously-existing topologies, its usage is limited to design improvement and refactoring phases rather than design. Although this limitation may inhibit practitioners from using our technology, the (anti-)patterns and algorithmic approaches elaborated in this paper help designers and implementors to develop the reasonably good-quality and ``quick" topologies upon which to use OSTIA for continuous improvement.

Third, OSTIA does offer essential insights to aid deployment as well (e.g., separating or \emph{clustering} complex portions of a topology so that they may run on dedicated infrastructure), however, our tool was not meant to be used as a system to aid planning and infrastructure design. Rather, as specified previously in the introduction, OSTIA was meant to evaluate and increase the quality of topologies \emph{before} they enter into operation since the continuous improvement cycles connected to operating the topology and learning from operation are often costly and still greatly inefficient. \comment{but we provided evidences that we exploit the operation data for formal verification and performance improvements. this paragraph is a bit vague, please edit}

Fourth, although we were able to discover a number of recurrent anti-patterns to be applied during OSTIA analysis, we were not able to implement all of them in practice and in a manner which allows to spot both the anti-pattern and any problems connected with it. For example, detecting the ``Cycle-in topology" is already possible, however, OSTIA would not allow designers to understand the consequence of the anti-pattern, i.e., where in the infrastructure do the cycles cause troubles. Also, there are several features that are currently under implementation but not released within the OSTIA codebase, for example, the ``Persistent Data" and the ``Topology Cascading" features.

In the future we plan to tackle the above limitations furthering our understanding of streaming design as well as the support OSTIA offers to designers during continuous architecting.



\section{Related Work}
\label{rw}
%\begin{itemize}
%\item mention DICE
%\item mention work by Len Bass on Big-Data
%\item other stuff on big data?
%\item feel free to extend this section with Previous work of course :)
%\end{itemize}

The work behind OSTIA stems from the EU H2020 Project called DICE\footnote{\url{http://www.dice-h2020.eu/}} where we are investigating the use of model-driven facilities to support the design and quality enhancement of Big-Data applications. In the context of DICE, much previous work has been done to support the design, development and deployment of Big-Data applications. For example, we have been developing a series of ad-hoc technological specifications, i.e., meta-model packages that contain all concepts, constructs and constraints needed to develop and operate Big-Data applications for the frameworks coded into the technological specifications. These specifications can be used, for example, to instantiate Big-Data components following standard Model-Driven procedures, without the need to learn said frameworks at all. OSTIA uses insights gained in developing and using said frameworks to apply consistency checks in the context of recovering Big-Data architectures, specifically, for Storm. Much similarly to the DICE effort, the IBM Stream Processing Language (SPL) initiative \cite{ibmspl} provides an implementation language specific to programming streams management (e.g., Storm jobs) and related reactive systems based on the Big-Data paradigm. Although SPL is specific to WebSphere and IBM technology, its attempt at providing a relatively abstract language to implement for streams management and processing is remarkably related to OSTIA, since one of its aims is to improve quality of streams management by direct codification of higher order concepts such as streams declarations.

In addition, there are several work close to OSTIA in terms of their foundations and type of support. 

First, from a quantitative perspective, much literature discusses quality analyses of Storm topologies, e.g., from a performance~\cite{perfbd} or reliability point of view \cite{bigdatareliab}. Said works use complex math-based approaches to evaluating a number of Big data architectures, their structure and general configuration. However, although novel, these approaches do not suggest any significant design improvement method or pattern to make the improvements \emph{deployable}. With OSTIA, we make available a tool that automatically elicits a Storm topology and, while doing so, analyses said topology to evaluate it against a number of consistency checks that make the topology consistent with the framework it was developed for (Storm, in our case). As previously introduced, a very trivial example of said checks consists in evaluating wether the topology is indeed a Directed-Acyclic-Graph (DAG), as per constraints of the Storm framework.  To the best of our knowledge, no such tool exists to date. \\
%\textbf{@Andrea,Pooyan: I think here could be a good idea to cite one or two of your previous works, e.g., with optimising Storm topologies}
Second, in our previous work, we proposed {\small \sf BO4CO} \cite{jamshidi-vldb}, an approach for locating optimal configurations using ideas of carefully choosing where to sample by sequentially reducing uncertainty in the response surface approximation in order to reduce the number of performance measurements. We have carried out extensive experiments with three different stream topologies running on Apache Storm. Experimental results demonstrate that {\small \sf BO4CO} outperforms the baselines in terms of distance to the optimum performance with at least an order of magnitude. 

Third, from a modelling perspective, approaches such as StormGen~\cite{stormgen} offer means to develop Storm topologies in a model-driven fashion using a combination of generative techniques based on XText and heavyweight (meta-)modelling, based on EMF, the standard Eclipse Modelling Framework Format. Although the first of its kind, StormGen merely allows the specification of a Storm topology, without applying any consistency checks or without offering the possibility to \emph{recover} said topology once it has been developed. By means of OSTIA, designers and developers can work hand in hand while refining their Storm topologies, e.g., as a consequence of verification or failed checks through OSTIA. As a consequence, tools such as StormGen can be used to assist the preliminary development of quick-and-dirty Storm topologies, e.g., to be further processed and improved with the aid of OSTIA.

<<<<<<< HEAD
Fourth, from a verification perspective, no previous effort tried yet to combine formal verification and architectural modelling of streaming topologies. Our attempt serves as a first rudimentary effort towards using complex and valuable verification approaches in combination with lightweight and agile DevOps inspired tools and approaches.
%...\\
%\textbf{@Marcello,Francesco: here we should probably elaborate on what kind of verification approach we are using and what other verifications may be done, e.g., using some related work at this point... e.g., is there any other verification attempt considering JSON as an interchange format? I would discuss these and compare them to OSTIA as a whole}
=======
\textbf{@Marcello,Francesco: here we should probably elaborate on what kind of verification approach we are using and what other verifications may be done, e.g., using some related work at this point... e.g., is there any other verification attempt considering JSON as an interchange format? I would discuss these and compare them to OSTIA as a whole}\\
Fourth, from a verification perspective, to the best of our knowledge, this represents the first attempt to build a formal model representing Storm topologies, and the first try in making a configurable model aiming at running verification tasks of non-functional properties for big data applications. While some works concentrate on exploiting big data technologies to speedup verification tasks~\cite{camilli2014}, others focus on the formalization of the specific framework, but remain application-independent, and their goal is rather to verify properties of the framework, such as reliability and load balancing~\cite{dicomputational}, or the validity of the messaging flow in MapReduce~\cite{yang2010formalizing}.

>>>>>>> origin/master

Finally, several deployment modelling technologies may be related to OSTIA since their role is to model the deployment structure represented by Big data architectures so that it can actually be deployed using compliant orchestrators. One such example is Celar\footnote{\url{https://github.com/CELAR/c-Eclipse}}, a deployment modelling technology based on the TOSCA OASIS Standard\footnote{\url{https://www.oasis-open.org/apps/org/workgroup/tosca/}}. Celar and related technologies (e.g., Alien4Cloud\footnote{\url{http://alien4cloud.github.io/}}) may be used in combination with OSTIA since their role is that of representing the infrastructure needed by modelled (Big data) applications so that they can be deployed. This representation is realised by means of infrastructure blueprints to be run by compliant orchestrators. The role of OSTIA in this scenario, is that of helping the quality refinement of an application topology to represent the very infrastructure needed for its run-time environment.



\section{Conclusion}
Applications that make heavy use of Big data application frameworks require intensive reasoning and continuous architecting of design aspects typically around the topology of the basic operations to be applied on the data. We set out to assist the continuous architecting of Big-Data streaming designs by OSTIA, a toolkit to assist designers and developers in this continuous architecting campaign. OSTIA helps designers and developers by recovering and analysing the architectural topology on-the-fly, assisting them in: (a) reasoning on the topological structure and how to refine it; (b) export the topological structure consistently with restrictions of their reference development framework so that further analysis (e.g., formal verification) may ensue. In addition, while performing on-the-fly architecture recovery, the analyses that OSTIA is able to apply focus on checking for the compliance to essential consistency rules specific to targeted big data frameworks. Finally, OSTIA allows to check whether the recovered topologies contain occurrences of key anti-patterns we observed in our own experience and previous work. By running a case-study with a partner organization, we observed that OSTIA assists designers and developers in establishing and continuously improving the quality of topologies behind their big data applications in multiple ways. We confirmed this result running OSTIA on several open-source applications featuring streaming technologies.
 
In the future we plan to further our understanding of the anti-patterns that may emerge across big data topologies, e.g., as discussed, by learning said anti-patterns by using graphs analysis techniques inherited from social-networks analysis. Also, we plan to expand OSTIA to support further technologies beyond the most common application framework for streaming, i.e., Storm. Finally, we plan to further evaluate OSTIA using more ad-hoc empirical evaluation in industry.
\label{conc}

%\clearpage
% conference papers do not normally have an appendix

% trigger a \newpage just before the given reference
% number - used to balance the columns on the last page
% adjust value as needed - may need to be readjusted if
% the document is modified later
%\IEEEtriggeratref{8}
% The triggered command can be changed if desired:
%\IEEEtriggercmd{\enlargethispage{-5in}}

% references section

% can use a bibliography generated by BibTeX as a .bbl file
% BibTeX documentation can be easily obtained at:
% http://www.ctan.org/tex-archive/biblio/bibtex/contrib/doc/
% The IEEEtran BibTeX style support page is at:
% http://www.michaelshell.org/tex/ieeetran/bibtex/
\balance

%% The Appendices part is started with the command \appendix;
%% appendix sections are then done as normal sections
%% \appendix

%% \section{}
%% \label{}

%% If you have bibdatabase file and want bibtex to generate the
%% bibitems, please use
%%
\footnotesize
  \bibliographystyle{elsarticle-num} 
  \bibliography{ostia}

%% else use the following coding to input the bibitems directly in the
%% TeX file.


\end{document}
\endinput
%%
%% End of file `elsarticle-template-num.tex'.
