Big data applications are rapidly gaining interest and momentum by small to big players on the market, even beyond the IT corner. Applications that may heavy use of Big data application frameworks require intensive reasoning on design aspects typically around the topology of basic operations to be applied while manipulating the data. 
In this paper we presented OSTIA, a tool to assist designers and developers in this reasoning campaign. OSTIA helps designers and developers improving the quality of their big data topologies by recovering and analysing the architectural topology on-the-fly. The analyses that OSTIA is able to apply focus on checking for the compliance to essential consistency rules specific to targeted big data frameworks. Also, OSTIA allows to check wether the recovered topologies contain occurrences of key anti-patterns we elicited and studied in our own experience and previous work. In addition, OSTIA allows to export the recovered topology using a common JSON format to encourage further verification using known verification and validation technology. 
By running a case-study with a partner organization, we observed that OSTIA assists designers and developers in establishing and improving the quality of topologies behind big data applications. 
In the future we plan to further our understanding of the anti-patterns that may emerge across big data topologies, e.g., learning said anti-patterns by using graphs analysis techniques inherited from social-networks analysis. Also, we plan to expand OSTIA to support further technologies beyond the most common application framework for streaming, i.e., Storm. Finally, we plan to further evaluate OSTIA using more ad-hoc empirical evaluation campaigns in industry.