%%\begin{itemize}
%%\item I would follow the path of the abstract, we should probably provide some numbers and info on storm
%%\item mind you we should stress on the innovative aspects of the paper and tech. there is nothing strictly related to it
%%\item we should comment on what could be done with OSTIA in combination with Eclipse Based tech.
%%\end{itemize}
%%%Big data architectures have been gaining momentum in the last few years. For example, Twitter uses complex Stream topologies featuring frameworks like Storm to analyse and learn trending topics from billions of tweets per minute. However, verifying the consistency of said topologies often requires de- ployment on multi-node clusters and can be expensive as well as time consuming. As an aid to designers and developers evaluating their Stream topologies at design-time, we developed OSTIA, that is, ?On-the-fly Storm Topology Inference Analysis?. OSTIA allows reverse-engineering of Storm topologies so that designers and developers may: (a) use previously existing model- driven verification&validation techniques on elicited models; (b) visualise and evaluate elicited models against consistency checks that would only be available at deployment and run-time. We illustrate the uses and benefits of OSTIA on three real-life industrial case studies.

Big data applications process vast amounts of data \cite{bdsurvey} for the purpose of gaining key business intelligence through complex analytics such as machine-learning \cite{ml4bd}. Said applications are receiving increased attention in the last few years given their ability to yield competitive advantage by direct investigation of user needs and trends hidden in the enormous quantities of data produced daily by the average internet user. Gartner predicts\footnote{\url{http://www.gartner.com/newsroom/id/2637615}} that said business intelligence and analytics applications will remain top focus for CIOs until at least 2017-2018. 
However, there are many costs and complexities behind harnessing said applications ranging from very high infrastructure costs to steep learning curves for the many frameworks (e.g., Apache Storm\footnote{\url{http://storm.apache.org/}}, Apache Spark\footnote{\url{http://spark.apache.org/}} or Hadoop\footnote{\url{https://hadoop.apache.org/}} to name a few) involved in designing and developing applications for Big Data.

In our own experience with designing and developing for Big Data, we observed that one such key complexity lies in evaluating architectures' effectiveness, that is, their deployability against maintained consistency with necessary Big data framework restrictions. Making sure this effectiveness is maintained before deployment means saving the time of running trial-and-error experiments while setting up expensive infrastructure needs as well as while fine-tuning complex framework setup options.

In an effort to aid designers and developers to jointly refine their Big data applications at design time, we developed OSTIA, that stands for: ``On-the-fly Static Topology Inference Analysis". OSTIA allows designers and developers to quickly start refining their application collaboratively by inferring the application topology though reverse-engineering and architecture recovery \cite{archrec}. OSTIA currently focuses on Storm, the most famous and established real-time stream processing Big data engine \cite{storm}. OSTIA hardcodes intimate knowledge on the development and framework dependence structure necessary for correct Storm topologies for two purposes: (a) realising a visual representation for said topologies; (b) export said topologies for further analysis.

This paper outlines OSTIA, elaborating its major usage scenarios, its benefits while properly discussing and addressing its limitations. Also, we evaluate OSTIA on three industrial case-studies from an open-source social-sensing application to show that OSTIA yields valuable design and development insights using an inference analysis of the three static topologies for said application. We conclude that OSTIA does in fact provide valuable insights for software engineers to increase quality of their Big data design and development featuring Storm.

The rest of the paper is structured as follows. Section \ref{ra} outlines our research problem, research questions and our approach at tackling them. Section \ref{rs} outlines OSTIA, discussing its typical usage scenarios and outlining the main benefits we perceived in using it. Section \ref{eval} elaborates further on the benefits by providing an actual evaluation of OSTIA using three cases from industrial practice. Section \ref{disc} discusses the results and evaluation, also outlining OSTIA limitations and potential threats to its validity. Finally, Sections \ref{rw} and \ref{conc} outline related work and conclude the paper.