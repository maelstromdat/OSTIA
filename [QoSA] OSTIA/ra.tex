%\begin{itemize}
%\item so we had a focus group to actually elaborate the aprroach 
%\item then we used explorative prototyping to elicit the initial version of the prototype and then refined that by means of case study, we could mention that we used ATC as an action-research source (this is what we are doing now internally in WP2 actually)
%\item ...
%\end{itemize}

The work we elaborated in this paper is stemming from the following research question:

\begin{center}
\emph{``Can we improve the structural quality of Big-Data streaming designs?"}
\end{center}

The results contained in this paper in response to this research question were initially elaborated within a free-form focus group \cite{focusgroup} involving three experienced practitioners and researchers on Big data streaming technology such as Storm. Following the focus group, through self-ethnography \cite{selfeth} and brainstorming we designed OSTIA and identified the series of consistency checks as well as anti-patterns that can now be evaluated through OSTIA while recovering an architectural representation for Storm topologies. The prototype thus developed, was refined incrementally through a series of case-studies \cite{casestudy} involving an industrial partner. Finally, we applied well established verification approaches to integrate the value and benefits behind using OSTIA.

\begin{itemize}
\item elaborate on the case-study partner
\item elaborate on the case at hand for that partner
\item elaborate on the origin of the application and how it uses social-sensor
\item also, elaborate on the case by NETF which starts and stems from KILLRWEATHER
\item should we elaborate on something else?
\end{itemize}
\textbf{TODO: @anyone, feel free to elaborate more!!}

