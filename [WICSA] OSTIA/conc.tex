%Applications that make heavy use of big data application frameworks require intensive reasoning of the system architecture. 
We set out to assist the continuous architecting of big data streaming designs by OSTIA, a toolkit to assist designers and developers to facilitate static analysis of the architecture and provide automated constraint verification in order to identify design anti-patterns and provide structural refactorings. OSTIA helps designers and developers by recovering and analysing the architectural topology on-the-fly, assisting them in: (a) reasoning on the topological structure and how to refine it; (b) export the topological structure consistently with restrictions of their reference development framework so that further analysis (e.g., formal verification) may ensue. In addition, while performing on-the-fly architecture recovery, the analyses that OSTIA is able to apply focus on checking for the compliance to essential consistency rules specific to targeted big data frameworks. Finally, OSTIA allows to check whether the recovered topologies contain occurrences of key anti-patterns. By running a case-study with a partner organization, we observed that OSTIA assists designers and developers in establishing and continuously improving the quality of topologies behind their big data applications. We confirmed this result running OSTIA on several open-source applications featuring streaming technologies. We released OSTIA as an open-source software~\cite{ostia}. % \url{https://github.com/maelstromdat/OSTIA}. 
 
In the future we plan to further elaborate the anti-patterns, 
%that may emerge across big data topologies 
exploiting graphs analysis techniques inherited from social-networks analysis. Also, we plan to expand OSTIA to support further technologies beyond the most common application framework for streaming, i.e., Storm. Finally, we plan to further evaluate OSTIA using more ad-hoc empirical evaluation in industry.

{\small\subsubsection*{Acknowledgment} The Authors' work is partially supported by the European Commission grant no. 644869 (EU H2020), DICE. Also, Damian's work is partially supported by the European Commission grant no. 610531 (FP7 ICT Call 10), SeaClouds.}