Big data applications are rapidly gaining interest and momentum by small to big players on the market, even beyond the IT corner. Applications that make heavy use of Big data application frameworks require intensive reasoning and continuous architecting of design aspects typically around the topology of the basic operations to be applied on the data. 

We set out to answer the following research question: ``How can we assist the continuous architecting of Big-Data streaming designs?" As an answer to this question this paper elaborates on OSTIA, a toolkit to assist designers and developers in this continuous architecting campaign. OSTIA helps designers and developers by recovering and analysing the architectural topology on-the-fly, assisting them in: (a) reasoning on the topological structure and how to refine it; (b) export the topological structure consistently with restrictions of their reference development framework so that further analysis (e.g., formal verification) may ensue. In addition, while performing on-the-fly architecture recovery, the analyses that OSTIA is able to apply focus on checking for the compliance to essential consistency rules specific to targeted big data frameworks. Finally, OSTIA allows to check wether the recovered topologies contain occurrences of key anti-patterns we observed in our own experience and previous work. 

By running a case-study with a partner organization, we observed that OSTIA assists designers and developers in establishing and continuously improving the quality of topologies behind their big data applications in multiple ways. We confirmed this result running OSTIA on several open-source applications featuring streaming technologies.
 
In the future we plan to further our understanding of the anti-patterns that may emerge across big data topologies, e.g., as discussed, by learning said anti-patterns by using graphs analysis techniques inherited from social-networks analysis. Also, we plan to expand OSTIA to support further technologies beyond the most common application framework for streaming, i.e., Storm. Finally, we plan to further evaluate OSTIA using more ad-hoc empirical evaluation in industry.