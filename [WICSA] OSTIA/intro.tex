%%\begin{itemize}
%%\item I would follow the path of the abstract, we should probably provide some numbers and info on storm
%%\item mind you we should stress on the innovative aspects of the paper and tech. there is nothing strictly related to it
%%\item we should comment on what could be done with OSTIA in combination with Eclipse Based tech.
%%\end{itemize}
%%%Big data architectures have been gaining momentum in the last few years. For example, Twitter uses complex Stream topologies featuring frameworks like Storm to analyse and learn trending topics from billions of tweets per minute. However, verifying the consistency of said topologies often requires de- ployment on multi-node clusters and can be expensive as well as time consuming. As an aid to designers and developers evaluating their Stream topologies at design-time, we developed OSTIA, that is, ?On-the-fly Storm Topology Inference Analysis?. OSTIA allows reverse-engineering of Storm topologies so that designers and developers may: (a) use previously existing model- driven verification&validation techniques on elicited models; (b) visualise and evaluate elicited models against consistency checks that would only be available at deployment and run-time. We illustrate the uses and benefits of OSTIA on three real-life industrial case studies.
%%%%%%
%%%%%% intro needs a bit of refinement with what we say in the title and a few more definitions should be included (e.g., about streaming and what it represents or why topologies ?are? the Big Data architecture)? Perhaps we should also increase the stress and focus on Quality and deployability aspects (i.e., the main topics of next year?s QoSA) in the intro, and how OSTIA aids at improving these aspects

Big data applications process large amounts of data for the purpose of gaining key business intelligence through complex analytics such as machine-learning \cite{bdsurvey, ml4bd}. These applications are receiving increased attention in the last years given their ability to yield competitive advantage by direct investigation of user needs and trends hidden in the enormous quantities of data produced daily by the average Internet user. According to Gartner\footnote{\url{http://www.gartner.com/newsroom/id/2637615}} business intelligence and analytics applications will remain a top focus for CIOs until at least 2017-2018.
However, the cost of ownership of the systems that process big data analytics are high due to high infrastructure costs, steep learning curves for the different frameworks involved in designing and developing the applications, such as Apache Storm\footnote{\url{http://storm.apache.org/}}, Apache Spark\footnote{\url{http://spark.apache.org/}} or Apache Hadoop\footnote{\url{https://hadoop.apache.org/}} and complexities in large-scale architectures.

In our own experience with designing and developing for big data, we observed that a key complexity lies in quickly and continuously evaluating the effectiveness of big-data architectures. Effectiveness, in big data terms, means being able to design, deploy, operate, refactor and then (re-)deploy architectures continuously and consistently with runtime restrictions imposed by the development frameworks. Storm, for example, requires the processing elements to represent a Directed-Acyclic-Graph (DAG). In toy topologies, such constraints can be effectively checked manually, however, when the number of nodes in such architectures increases to real-life industrial scale architectures, it is enormously difficult to verify even these ``simple" structural DAG constraints.
%\textbf{TODO: can we add an example of said consistency checks/issues?} \\
We argue that this effectiveness can be maintained starting from design time, by enacting a continuous architecting of big data architectures consistently with a DevOps organisational structure \cite{ossslr,devops}. Supporting this continuous architecting exercise, eases the (re-)deployability of big data architectures and saves the effort to run trial-and-error experiments on expensive infrastructure.

To sustain this argument, we developed OSTIA, that stands for: ``On-the-fly Static Topology Inference Analysis". OSTIA allows designers and developers to infer the application architecture through on-the-fly reverse-engineering and architecture recovery \cite{archrec}. During this inference step, OSTIA analyses the architecture to verify whether it is consistent with development restrictions and/or deployment constraints of the underlying development frameworks (e.g., DAG constraints). To do so, OSTIA hardcodes intimate knowledge on the streaming development framework (Storm, in our case) and its dependence structure in the form of a meta-model \cite{mda}. This knowledge is necessary to make sure that elicited topologies are correct, so that models may be used in at least four scenarios: (a) realising an exportable visual representation of the developed topologies; (b) verifying structural constraints on topologies that would only become evident during infrastructure setup or runtime operation; (c) verifying the topologies against anti-patterns \cite{patternoriented2000} that may lower performance and limit deployability/executability; (d) finally, use topologies for further analysis, e.g., through model verification \cite{icsoft}. In an effort to offer said support in a DevOps fashion, OSTIA was engineered to act as an architecture recovery mechanism that closes the feedback loop between operational data architectures (Ops phase) and their refactoring phase (Dev phase). 

Currently, OSTIA focuses on Apache Storm, i.e., one of the most famous and established real-time stream processing engine \cite{storm, toshniwal2014storm}. The core element of Storm, is called \emph{topology}, which represents the architecture of the processing components of the application (from now we use topology and architecture interchangeably).

This paper outlines OSTIA, elaborating major usage scenarios, benefits and limitations. Also, we evaluate OSTIA using case-study research and formal validation to conclude that OSTIA does in fact provide valuable insights for continuous architecting.

The rest of the paper is structured as follows. Section \ref{ra} outlines our research design. Sections \ref{rs}, \ref{sec:anti-pattern} and \ref{algo} describe OSTIA, discussing the (anti-)patterns it supports and its usage scenarios. Section \ref{eval} evaluates OSTIA while Section \ref{disc} discusses results and evaluation outlining OSTIA limitations and threats to validity. Finally, Sections \ref{rw} and \ref{conc} report related work and conclude the paper.
