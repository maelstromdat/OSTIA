<<<<<<< HEAD
\newcommand{\I}{\mathcal{I}}
\newcommand{\M}{\mathcal{M}}
\newcommand{\timestr}{\mathcal{T}}

=======
>>>>>>> 915d87a8f9df7f67cfd959767461c7f00a2f4d2b
\newcommand{\D}{\mathcal{D}}
\newcommand{\C}{\mathcal{C}}

%old, compatibility reasons
\newcommand{\U}{\mathbf{U}}
\newcommand{\Snc}{\mathbf{S}}
\newcommand{\T}{\mathbf{T}}
\newcommand{\R}{\mathbf{R}}

\newcommand{\Nat}{\mathbb{N}}
\newcommand{\Z}{\mathbb{Z}}
\newcommand{\Real}{\mathbb{R}}
\newcommand{\Q}{\mathbb{Q}}

%old, compatibility reasons

\newcommand{\X}[1]{\mathbf{X}\left(#1\right)}
\newcommand{\Y}[1]{\mathbf{Y}\left(#1\right)}

\newcommand{\X}{\mathbf{X}}
\newcommand{\Y}{\mathbf{Y}}


\newcommand{\Zed}{\mathbf{Z}}
\newcommand{\Lng}{\mathscr {L}}
\newcommand{\iFF}{\Leftrightarrow}
\newcommand{\niFF}{\nLeftrightarrow}
\newcommand{\SNC}{{\mathcal S}}
\newcommand{\TRG}{{\mathcal T}}
\newcommand{\zot}{$\mathds{Z}$ot}
%old, compatibility reasons
\newcommand{\G}[1]{\mathbf{G}\left(#1\right)}
\newcommand{\F}[1]{\mathbf{F}\left(#1\right)}
%\newcommand{\Q}{\mathbb{Q}}

\newcommand{\triple}[3]{(#1, #2, #3)}
\newcommand{\pair}[2]{(#1, #2)}
\newcommand{\siff}{\Leftrightarrow}
\newcommand{\A}{\mathcal{A}}
\newcommand{\aX}{\mathrm{X}}
\newcommand{\aY}{\mathrm{Y}}
\newcommand{\x}{\mathbf{x}}
\newcommand{\eqdef}{\stackrel{\mbox{\begin{tiny}def\end{tiny}}}{=}} % =def=
\newcommand{\iFFdef}{\stackrel{\mbox{\begin{tiny}def\end{tiny}}}{\iFF}}
% =def=
\newcommand{\step}[1]{\xrightarrow{#1}}

\newcommand{\pspace}{\textsc{PSpace}}


\makeatletter
\def\Eqlfill@{\arrowfill@\Relbar\Relbar\Relbar}
\newcommand{\longmodels}[1][]{\,|\!\!\!\ext@arrow 0359\Eqlfill@{#1}}
\makeatother

\newcommand{\symodels}{\longmodels{\mbox{\it{\tiny sym}}}}

\newcommand{\intervaLii}[2]{[#1,#2]}
\newcommand{\intervaLie}[2]{[#1,#2)}
\newcommand{\intervaLee}[2]{(#1,#2)}

\newcommand{\interval}[2]{\langle #1,#2 \rangle}

\newcommand{\set}[1]{\{ #1 \}}

\newcommand{\tsys}[1]{\mathcal{S}(#1)}


\newcommand{\lapp}[1]{\lfloor #1 \rfloor}
\newcommand{\happ}[1]{\lceil #1 \rceil}


\newcommand{\first}[2]{(H_{#1}\vee L_{#1}) \wedge(\neg(H_{#1}\vee L_{#1}) \Snc (#2))}


<<<<<<< HEAD

\newcommand{\pname}[1]{\ensuremath{\textit{#1}}}
\newcommand{\on}{\pname{on}}
\newcommand{\off}{\pname{off}}
\newcommand{\lon}{\pname{l}}
\newcommand{\test}{\pname{test}}
\newcommand{\resetc}{\pname{reset-c}}
\newcommand{\turnoff}{\pname{turnoff}}




\newcommand{\edge}[1]{\texttt{#1}}
\newcommand{\enabled}[1]{\texttt{e}_{#1}}

\newcommand{\visit}[1]{\mathit{visit}(#1)}
\newcommand{\inv}[1]{\mathit{inv}(q_{#1})}


\newcommand{\intg}[1]{\lfloor#1\rfloor}
\newcommand{\fract}[1]{\mathit{frac(#1)}}
=======
\newcommand{\todoMR}[2]{\todo[color=red!50,#1]{\textbf{MR:}#2}}
\newcommand{\todoPL}[2]{\todo[color=yellow!50,#1]{\textbf{PL:}#2}}
\newcommand{\todoMB}[2]{\todo[color=green!50,#1]{\textbf{MB:}#2}}


\def\notRaisingEdge{\mathrel{\NOTRaisingEdge}}
\def\NOTRaisingEdge{{%
    \setbox0\hbox{\RaisingEdge}%
    \rlap{\hbox to \wd0{\hss/\hss}}\box0
}}


\def\notFallingEdge{\mathrel{\NOTFallingEdge}}
\def\NOTFallingEdge{{%
    \setbox0\hbox{\FallingEdge}%
    \rlap{\hbox to \wd0{\hss/\hss}}\box0
}}


\def\notShortPulseHigh{\mathrel{\NOTShortPulseHigh}}
\def\NOTShortPulseHigh{{%
    \setbox0\hbox{\ShortPulseHigh}%
    \rlap{\hbox to \wd0{\hss/\hss}}\box0
}}


\def\notShortPulseLow{\mathrel{\NOTShortPulseLow}}
\def\NOTShortPulseLow{{%
    \setbox0\hbox{\ShortPulseLow}%
    \rlap{\hbox to \wd0{\hss/\hss}}\box0
}}


%% Some definitions needed to use MnSymbol \mediamond and \medsquare
  \DeclareFontFamily{U}  {MnSymbolA}{}
  \DeclareSymbolFont{MnSyA}         {U}  {MnSymbolA}{m}{n}
  \SetSymbolFont{MnSyA}       {bold}{U}  {MnSymbolA}{b}{n}
\DeclareFontShape{U}{MnSymbolA}{m}{n}{
    <-6>  MnSymbolA5
   <6-7>  MnSymbolA6
   <7-8>  MnSymbolA7
   <8-9>  MnSymbolA8
   <9-10> MnSymbolA9
  <10-12> MnSymbolA10
  <12->   MnSymbolA12}{}
\DeclareFontShape{U}{MnSymbolA}{b}{n}{
    <-6>  MnSymbolA-Bold5
   <6-7>  MnSymbolA-Bold6
   <7-8>  MnSymbolA-Bold7
   <8-9>  MnSymbolA-Bold8
   <9-10> MnSymbolA-Bold9
  <10-12> MnSymbolA-Bold10
  <12->   MnSymbolA-Bold12}{}
  \DeclareMathSymbol{\leftfilledspoon}{\mathrel}{MnSyA}{114}
  \DeclareMathSymbol{\leftspoon}{\mathrel}{MnSyA}{106}
  \DeclareMathSymbol{\upfilledspoon}{\mathrel}{MnSyA}{113}
  \DeclareMathSymbol{\upspoon}{\mathrel}{MnSyA}{105}
  \DeclareMathSymbol{\leftfree}{\mathrel}{MnSyA}{130}

  \DeclareFontFamily{U}  {MnSymbolB}{}
  \DeclareSymbolFont{MnSyB}         {U}  {MnSymbolB}{m}{n}
  \SetSymbolFont{MnSyB}       {bold}{U}  {MnSymbolB}{b}{n}
\DeclareFontShape{U}{MnSymbolB}{m}{n}{
    <-6>  MnSymbolB5
   <6-7>  MnSymbolB6
   <7-8>  MnSymbolB7
   <8-9>  MnSymbolB8
   <9-10> MnSymbolB9
  <10-12> MnSymbolB10
  <12->   MnSymbolB12}{}
\DeclareFontShape{U}{MnSymbolB}{b}{n}{
    <-6>  MnSymbolB-Bold5
   <6-7>  MnSymbolB-Bold6
   <7-8>  MnSymbolB-Bold7
   <8-9>  MnSymbolB-Bold8
   <9-10> MnSymbolB-Bold9
  <10-12> MnSymbolB-Bold10
  <12->   MnSymbolB-Bold12}{}
  \DeclareMathSymbol{\nleftfilledspoon}{\mathrel}{MnSyB}{114}


%\newcommand{\rest}[1]{\ensuremath{\overrightarrow{#1}}}
%\newcommand{\fst}[1]{\ensuremath{|_{#1}}}
\newcommand{\rest}[1]{\ensuremath{\stackrel{\leftfree}{#1}}}
\newcommand{\fst}[1]{\ensuremath{\upfilledspoon_{#1}}}
\newcommand{\allup}[1]{\ensuremath{\stackrel{\leftfilledspoon}{#1}}}
\newcommand{\alldn}[1]{\ensuremath{\stackrel[\leftfilledspoon]{}{#1}}}


\newcommand{\Hg}[1]{\ensuremath{\textrm{\RaisingEdge}_{#1}}}
\newcommand{\notHg}[1]{\ensuremath{\notRaisingEdge_{#1}}}
\newcommand{\Lw}[1]{\ensuremath{\textrm{\FallingEdge}_{#1}}}
\newcommand{\notLw}[1]{\ensuremath{\notFallingEdge_{#1}}}
\newcommand{\Sup}[1]{\ensuremath{\textrm{\ShortPulseHigh}_{#1}}}
\newcommand{\notSup}[1]{\ensuremath{\notShortPulseHigh_{#1}}}
\newcommand{\Sdn}[1]{\ensuremath{\textrm{\ShortPulseLow}_{#1}}}
\newcommand{\notSdn}[1]{\ensuremath{\notShortPulseLow_{#1}}}

\newcommand{\qtlHg}[1]{\ensuremath{e^{u}_{#1}}}
\newcommand{\notqtlHg}[1]{\ensuremath{\neg eUp_{#1}}}
\newcommand{\qtlLw}[1]{\ensuremath{e^{d}_{#1}}}
\newcommand{\notqtlLw}[1]{\ensuremath{\neg eDn_{#1}}}
\newcommand{\qtlSup}[1]{\ensuremath{s^{u}_{#1}}}
\newcommand{\notqtlSup}[1]{\ensuremath{\neg s^{u}_{#1}}}
\newcommand{\qtlSdn}[1]{\ensuremath{s^{d}_{#1}}}
\newcommand{\notqtlSdn}[1]{\ensuremath{\neg s^{u}_{#1}}}


\DeclareRobustCommand{\shortf}[3] % operator_name, subscript, arguments
{\ensuremath{%
	 % if subscript argument (#3) is nonempty, add superscript appropriately
%	 \ifthenelse{\not \equal{#3}{} }  {{#1}_{#2}^{#3}}  {{#1}_{#2}}
	 \ifthenelse{\not \equal{#3}{} }  {\stackrel{#2}{#1}_{#3}}  {\stackrel{#2}{#1}}
  }%
}%

\def\udsh{\mathrel{\rotatebox[origin=c]{90}{$\dlsh$}}}
\def\dush{\mathrel{\rotatebox[origin=c]{-90}{$\Lsh$}}}

\DeclareRobustCommand{\befUnowD}[2]{\shortf{\udsh}{#1}{#2}}
\DeclareRobustCommand{\befDnowU}[2]{\shortf{\dush}{#1}{#2}}
\DeclareRobustCommand{\notbefUnowD}[2]{\shortf{\not \udsh}{#1}{#2}}
\DeclareRobustCommand{\notbefDnowU}[2]{\shortf{\not \dush}{#1}{#2}}
\DeclareRobustCommand{\nowonU}[2]{\shortf{\Rsh}{#1}{#2}}
\DeclareRobustCommand{\nowonD}[2]{\shortf{\drsh}{#1}{#2}}
\DeclareRobustCommand{\notnowonU}[2]{\shortf{\not \Rsh}{#1}{#2}}
\DeclareRobustCommand{\notnowonD}[2]{\shortf{\not \drsh}{#1}{#2}}

%\newcommand{\befUnowD}[2]{\shortf{\mathtt{lowEdge}}{#1}{#2}}
%\newcommand{\befDnowU}[2]{\shortf{\mathtt{highEdge}}{#1}{#2}}
%\newcommand{\nowonU}[1]{\shortf{\mathtt{up}}{#1}{}}
%\newcommand{\nowonD}[1]{\shortf{\mathtt{down}}{#1}{}}
\DeclareRobustCommand{\UbefD}[2]{\mathtt{up}_{{#1}\curvearrowright{#2}}}
\DeclareRobustCommand{\DafterU}[2]{\mathtt{down}_{{#2}\curvearrowleft{#1}}}
\DeclareRobustCommand{\orig}{\ensuremath{\mathrm{O}}}

\newcommand{\genconstr}[1]{\ensuremath{\mathtt{genconstr_{#1}}}}
\newcommand{\init}[1]{\ensuremath{\mathtt{init_{#1}}}}

\newcommand{\myitem}[1]{$\bullet \quad$ #1:~\hspace{6pt}~}
%\newcommand{\sig}{\ensuremath{\zeta}}
\newcommand{\sig}{\ensuremath{M}}
\newcommand{\m}[1]{\ensuremath{#1_{\text{CLTL}}}}
\newcommand{\rmap}[2]{\ensuremath{r_{#1}(#2)}}
\newcommand{\rmapinv}[2]{\ensuremath{r^{-1}_{#1}(#2)}}


%% Generic temporal operator, displayed in the form name_{subscript}(arguments)
\DeclareRobustCommand{\LogOp}[3] % operator_name, subscript, arguments
{\ensuremath{%
	 % if subscript argument (#2) is nonempty, add subscript appropriately
	 \ifthenelse{\not \equal{#2}{}}  {\mathbf{#1}_{#2}}  {\mathbf{#1}}
	 % if operator argument is nonempty write ( #3 )
	 \ifthenelse{\not \equal{#3}{}}  {\!\left({#3}\right)} {}
	 % else: just write the operator's name
  }%
}%

\DeclareRobustCommand{\LogOpPast}[3] % operator_name, subscript, arguments
{\ensuremath{%
	 % if subscript argument (#2) is nonempty, add subscript appropriately
	 \ifthenelse{\not \equal{#2}{}}  {\overleftarrow{\mathbf{#1}}_{#2}}  {\mathbf{#1}}
	 % if operator argument is nonempty write ( #3 )
	 \ifthenelse{\not \equal{#3}{}}  {\!\left({#3}\right)} {}
	 % else: just write the operator's name
  }%
}%

\DeclareRobustCommand{\Ultl}[1]{\LogOp{U}{#1}{}}
\DeclareRobustCommand{\Sltl}[1]{\LogOp{S}{#1}{}}
\DeclareRobustCommand{\Tltl}[1]{\LogOp{T}{#1}{}}
\DeclareRobustCommand{\Rltl}[1]{\LogOp{R}{#1}{}}
\DeclareRobustCommand{\Xltl}[1]{\LogOp{X}{}{#1}}
\DeclareRobustCommand{\Yltl}[1]{\LogOp{Y}{}{#1}}
\DeclareRobustCommand{\Gltl}[2]{\LogOp{G}{#1}{#2}}
\DeclareRobustCommand{\Fltl}[2]{\LogOp{F}{#1}{#2}}
\DeclareRobustCommand{\Pltl}[2]{\LogOp{P}{#1}{#2}}
\DeclareRobustCommand{\Pn}[2]{\LogOp{P}{#1}{#2}}
\DeclareRobustCommand{\Cn}[2]{\LogOp{C}{#1}{#2}}
\DeclareRobustCommand{\PnP}[2]{\LogOpPast{P}{#1}{#2}}
\DeclareRobustCommand{\CnP}[2]{\LogOpPast{C}{#1}{#2}}

\DeclareRobustCommand{\LogOpInfix}[4] % operator_name, subscript, arguments
{\ensuremath{%
	 % if first argument (#3) is nonempty, add first argument appropriately
	 \ifthenelse{\not \equal{#3}{}}  {#3}  {}
	 % if subscript argument (#2) is nonempty, add subscript appropriately
	 \ifthenelse{\not \equal{#2}{}}  {\mathbf{#1}_{#2}}  {\mathbf{#1}}
	 % if second argument is nonempty write it
	 \ifthenelse{\not \equal{#4}{}}  {#4} {}
	 % else: just write the operator's name
  }%
}%

\DeclareRobustCommand{\Uinfix}[3]{\LogOpInfix{U}{#1}{#2}{#3}}
\DeclareRobustCommand{\Sinfix}[3]{\LogOpInfix{S}{#1}{#2}{#3}}
\DeclareRobustCommand{\Tinfix}[3]{\LogOpInfix{T}{#1}{#2}{#3}}
\DeclareRobustCommand{\Rinfix}[3]{\LogOpInfix{R}{#1}{#2}{#3}}

\newcommand{\upn}[3]{\ensuremath{\mathtt{up}^{#1}_{#2}(#3)}}
\newcommand{\upSubfn}[3]{\ensuremath{\mathtt{upSub}^{#1}_{#2}(#3)}}
\newcommand{\nPspikes}[2]{\ensuremath{\overleftarrow{\mathtt{nspikes}}_{#1}(#2)}}
\newcommand{\upPn}[3]{\ensuremath{\overleftarrow{\mathtt{up}}^{#1}_{#2}(#3)}}
\newcommand{\thenY}[3]{\ensuremath{\mathtt{seq}^{#2}_{#3}(\rightsquigarrow #1)}}
\newcommand{\XthenY}[4]{\ensuremath{\mathtt{seq}^{#3}_{#4}(#1 \rightsquigarrow #2)}}
\newcommand{\XthenYthenZ}[5]{\ensuremath{\mathtt{seq}^{#4}_{#5}(#1 \rightsquigarrow #2 \rightsquigarrow #3)}}
\newcommand{\nowXthenY}[2]{\ensuremath{\mathtt{seqU}(#1 \rightsquigarrow #2)}}
\newcommand{\nowXthenYthenZ}[3]{\ensuremath{\mathtt{seqU}(#1 \rightsquigarrow #2 \rightsquigarrow #3)}}
\newcommand{\nowXthenYP}[2]{\ensuremath{\overleftarrow{\mathtt{seqU}}(#1 \leftsquigarrow #2)}}
\newcommand{\nowXthenYthenZP}[3]{\ensuremath{\overleftarrow{\mathtt{seqU}}(#1 \leftsquigarrow #2 \leftsquigarrow #3)}}
\newcommand{\XthenYP}[3]{\ensuremath{\overleftarrow{\mathtt{seq}}^{#3}(#1 \leftsquigarrow #2)}}
\newcommand{\XthenYthenZP}[4]{\ensuremath{\overleftarrow{\mathtt{seq}}^{#4}(#1 \leftsquigarrow #2 \leftsquigarrow #3)}}


\newcommand{\pname}[1]{\ensuremath{\textit{#1}}}
\newcommand{\on}{\pname{on}}
\newcommand{\off}{\pname{off}}
\newcommand{\lon}{\pname{l}}
\newcommand{\test}{\pname{test}}
\newcommand{\resetc}{\pname{reset-c}}
\newcommand{\turnoff}{\pname{turnoff}}
>>>>>>> 915d87a8f9df7f67cfd959767461c7f00a2f4d2b
