%\begin{itemize}
%\item mention DICE
%\item mention work by Len Bass on Big-Data
%\item other stuff on big data?
%\item feel free to extend this section with Previous work of course :)
%\end{itemize}

The work behind OSTIA stems from the EU H2020 Project called DICE\footnote{\url{http://www.dice-h2020.eu/}} where we are investigating the use of model-driven facilities to support the design and quality enhancement of Big-Data applications. In the context of DICE, much previous work has been done to support the design, development and deployment of Big-Data applications. For example, we have been developing a series of ad-hoc technological specifications, i.e., meta-model packages that contain all concepts, constructs and constraints needed to develop and operate Big-Data applications for the frameworks coded into the technological specifications. These specifications can be used, for example, to instantiate Big-Data components following standard Model-Driven procedures, without the need to learn said frameworks at all. OSTIA uses insights gained in developing and using said frameworks to apply consistency checks in the context of recovering Big-Data architectures, specifically, for Storm. Much similarly to the DICE effort, the IBM Stream Processing Language (SPL) initiative \cite{ibmspl} provides an implementation language specific to programming streams management (e.g., Storm jobs) and related reactive systems based on the Big-Data paradigm. Although SPL is specific to WebSphere and IBM technology, its attempt at providing a relatively abstract language to implement for streams management and processing is remarkably related to OSTIA, since one of its aims is to improve quality of streams management by direct codification of higher order concepts such as streams declarations.

In addition, there are several work close to OSTIA in terms of their foundations and type of support. 

First, from a quantitative perspective, much literature discusses quality analyses of Storm topologies, e.g., from a performance~\cite{perfbd} or reliability point of view \cite{bigdatareliab}. Said works use complex math-based approaches to evaluating a number of Big data architectures, their structure and general configuration. However, although novel, these approaches do not suggest any significant design improvement method or pattern to make the improvements \emph{deployable}. With OSTIA, we make available a tool that automatically elicits a Storm topology and, while doing so, analyses said topology to evaluate it against a number of consistency checks that make the topology consistent with the framework it was developed for (Storm, in our case). As previously introduced, a very trivial example of said checks consists in evaluating wether the topology is indeed a Directed-Acyclic-Graph (DAG), as per constraints of the Storm framework.  To the best of our knowledge, no such tool exists to date. \\
%\textbf{@Andrea,Pooyan: I think here could be a good idea to cite one or two of your previous works, e.g., with optimising Storm topologies}

Second, in our previous work, we proposed {\small \sf BO4CO} \cite{jamshidi-vldb}, an approach for locating optimal configurations using ideas of carefully choosing where to sample by sequentially reducing uncertainty in the response surface approximation in order to reduce the number of performance measurements. We have carried out extensive experiments with three different stream topologies running on Apache Storm. Experimental results demonstrate that {\small \sf BO4CO} outperforms the baselines in terms of distance to the optimum performance with at least an order of magnitude. 

Third, from a modelling perspective, approaches such as StormGen~\cite{stormgen} offer means to develop Storm topologies in a model-driven fashion using a combination of generative techniques based on XText and heavyweight (meta-)modelling, based on EMF, the standard Eclipse Modelling Framework Format. Although the first of its kind, StormGen merely allows the specification of a Storm topology, without applying any consistency checks or without offering the possibility to \emph{recover} said topology once it has been developed. By means of OSTIA, designers and developers can work hand in hand while refining their Storm topologies, e.g., as a consequence of verification or failed checks through OSTIA. As a consequence, tools such as StormGen can be used to assist the preliminary development of quick-and-dirty Storm topologies, e.g., to be further processed and improved with the aid of OSTIA.

\textbf{@Marcello,Francesco: here we should probably elaborate on what kind of verification approach we are using and what other verifications may be done, e.g., using some related work at this point... e.g., is there any other verification attempt considering JSON as an interchange format? I would discuss these and compare them to OSTIA as a whole}\\
Fourth, from a verification perspective, to the best of our knowledge, this represents the first attempt to build a formal model representing Storm topologies, and the first try in making a configurable model aiming at running verification tasks of non-functional properties for big data applications. While some works concentrate on exploiting big data technologies to speedup verification tasks~\cite{camilli2014}, others focus on the formalization of the specific framework, but remain application-independent, and their goal is rather to verify properties of the framework, such as reliability and load balancing~\cite{dicomputational}, or the validity of the messaging flow in MapReduce~\cite{yang2010formalizing}.


Finally, several deployment modelling technologies may be related to OSTIA since their role is to model the deployment structure represented by Big data architectures so that it can actually be deployed using compliant orchestrators. One such example is Celar\footnote{\url{https://github.com/CELAR/c-Eclipse}}, a deployment modelling technology based on the TOSCA OASIS Standard\footnote{\url{https://www.oasis-open.org/apps/org/workgroup/tosca/}}. Celar and related technologies (e.g., Alien4Cloud\footnote{\url{http://alien4cloud.github.io/}}) may be used in combination with OSTIA since their role is that of representing the infrastructure needed by modelled (Big data) applications so that they can be deployed. This representation is realised by means of infrastructure blueprints to be run by compliant orchestrators. The role of OSTIA in this scenario, is that of helping the quality refinement of an application topology to represent the very infrastructure needed for its run-time environment.

